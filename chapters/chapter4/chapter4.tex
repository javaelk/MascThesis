% UW Thesis Template for LaTeX 
% Last Updated July 8, 2009 by Stephen Carr, IST Client Services
% FOR ASSISTANCE, please send mail to rt-IST-CSmathsci@ist.uwaterloo.ca

% Effective October 2006, the University of Waterloo 
% requires electronic thesis submission. See the UW thesis regulations at
% http://www.grad.uwaterloo.ca/Thesis_Regs/thesistofc.asp.
% However, many faculties/departments also require one or more printed
% copies. This template attempts to satisfy requirements for both types of output. 
% It is based on the standard "report" document class. The "book" document 
% class can be substituted if you have a very large, multi-part thesis.

% DISCLAIMER
% To the best of our knowledge, this template satisfies the current UW requirements.
% However, it is your responsibility to assure that you have met all 
% requirements of the University and your particular department.
% Many thanks to Colin Alie for assistance in preparing this updated template.

% -----------------------------------------------------------------------

% By default, output is produced that is geared toward generating a PDF 
% version optimized for viewing on an electronic display, including 
% hyperlinks within the PDF.
 
% E.g. to process a thesis called "mythesis.tex" based on this template, run:

% pdflatex mythesis	-- first pass of the pdflatex processor
% bibtex mythesis	-- generates bibliography from .bib data file(s) 
% pdflatex mythesis	-- fixes cross-references, bibliographic references, etc
% pdflatex mythesis	-- fixes cross-references, bibliographic references, etc

% If you use the recommended LaTeX editor, Texmaker, you would open the mythesis.tex
% file, then click the pdflatex button. Then run BibTeX (under the Tools menu).
% Then click the pdflatex button two more times. If you have an index as well,
% you'll need to run MakeIndex from the Tools menu as well, before running pdflatex
% the last two times.

% N.B. The "pdftex" program allows graphics in the following formats to be
% included with the "\includegraphics" command: PNG, PDF, JPEG, TIFF
% Tip 1: Generate your figures and photos in the size you want them to appear
% in your thesis, rather than scaling them with \includegraphics options.
% Tip 2: Any drawings you do should be in scalable vector graphic formats:
% SVG, PNG, WMF, EPS and then converted to PNG or PDF, so they are scalable in
% the final PDF as well.
% Tip 3: Photographs should be cropped and compressed so as not to be too large.

% To create a PDF output that is optimized for double-sided printing: 
%
% 1) comment-out the \documentclass statement in the preamble below, and
% un-comment the second \documentclass line.
%
% 2) change the value assigned below to the boolean variable
% "ElectronicVersion" from "true" to "false".

% --------------------- Start of Document Preamble -----------------------

% Specify the document class, default style attributes, and page dimensions
% For hyperlinked PDF, suitable for viewing on a computer, use this:
%----------------------------------------------------------------------
% MAIN BODY
%----------------------------------------------------------------------
% Because this is a short document, and to reduce the number of files
% needed for this template, the chapters are not separate
% documents as suggested above, but you get the idea. If they were
% separate documents, they would each start with the \chapter command, i.e, 
% do not contain \documentclass or \begin{document} and \end{document} commands.
%======================================================================
\reversemarginpar
\setlength{\marginparwidth}{2cm}
\chapter{Modelling Goals}
%======================================================================
We use exiting goal modelling research to help us identify most apporapriate
metrics for selecting test cases. Though Software Test Selection 
\newpage
%----------------------------------------------------------------------
\section{Software Test Selection Goals}
%----------------------------------------------------------------------

\underline{why we talk about goals, different goals lead to differnt
techniques}Software Test Selection may be
performed for vaiours goals.Differences in underlying goals lead regresstion test selection
techniques to distinctly different results in test selection\cite{DBLP:journals/tse/RothermelH96}. 

Consider a conintuous integration development environment in which new or
changed code are integrated into existing code frequenly and an automated build
task is schedule nightly. In this case, test selection may be used to select a
small subset of existing test suite to test newly introduced or change code
only. Ideally , this selected subset is much smaller than the complete test
suite. If the new or changed code is to implmenent a new requirement or to
implment change to existing requirement, test selection may be used to select a
subset to test new or changed requirements.If the new or changed code is to
implmenent a fix of a defect, test selection may be used to select a subset to
test the fix. 

combination of multiple goals. conflict goals.




\cite{non-functional requirements in software engineering}
softgoal interdependency graphs

unlike goals in traditional problem-solving and planning frameworks,
non-functional requirmenents can rarely be said to be ``accomplished'' or
``satisfied'' in a clear-cut sense. Instead, differnet design decisions
contribute positively or negatively, and with variying degreees, towards
attaining a particular non-functional requirement. Accordingly, we speak of the
satisficing of a softgoals to suggest that generated software is expected to
satisfy NFRs within acceptable limits, rather than absolutely.


offspring can contribute fully or partially , and positiviely or negatively,
towards the satisficing of the parent.





\newpage
A very important feature of our framework are the catalogues of test
selection goals and test selection techniques based on research literautres to date. We
created two kinds of catalogues in our test selection framework. First, we
surveryed existing software test selection,prioritization and minimization papers and created a catalog of software test selection goals. Then, we
systematically organie test selection techniques discuss in the literature.
These techniques are intend to help meet selection goals. With these two
catalogues, we created a SIG to capture the interdependecies
between software test selection goals and relationships between selection goals
and selection techniques. These two catalogues not only the first step for test engineers to pick the right selection techniques,
but also are the fundmental knowledge base for test engineers to understand what
other selection options are available out there. In this chapter, we will discuss in details how
this SIG is created.

%document how I come up with the graph. how the survey was done ans what
% information is collection and compared sysmetically
\subsection{Acquiring and cataloguing Software Test Selection goals}
We followed the decomposing process in \cite{NFR} to decompose our top level
goal. The initial top level softgoal is pretty broad,it needs to be broken down
into smaller subgoals, until a metrics and a selection technqiue can be
found.Together, subgoals ``satisfice'' (should meet) the higher level softgoals.

The catalogues of test selection goals and selection technqiues are composed
from many sources. Once source is  that we have survey test selection papers
published in the major software engineering venue since year 1995, especially in
software testing and softwar maintenance literature. Since there are some
overlaps between test selection and prioritization,minimization, we also extend our survery to prioritization and minimization in some cases. A
complete list of paper survery to build the catagloue is in appendix A.
In surveying test selection releated literatures, we extract information deplict
in figure xx for each paper. Some of the information are not explictly mentioned
in the paper. For example,  most of the tesst optimization research were
performed with one or several goals in mind, but not all goals are specified
explictily. In that case, we have documented the goals for the paper  based on
our understanding of the research. 

\underline{add diagram here goal-data-metrics-evaluation metrics etc. on
whiteboard}
\subsection{Identified test selection goals}


\subsection{summary}
Some selection goals and techniques are also come from the knowledge in the
industry. As this framework is used in practice, test engineers can add
additional goals and techniques to the catalogues in the framework. Knowledge
and experience from wide range of industry settings can be recorded in the
framework and made available for sharing. Test engineers and researcher would
have access to selection goals and techniques beyond their immediate domain ,
and can adapt them to meet the needs for particular domain or project.




In addition, some of the goals are ambigous or not granaluar enough. 


%----------------------------------------------------------------------
% END MATERIAL
%----------------------------------------------------------------------
% B I B L I O G R A P H Y
% -----------------------

% The following statement selects the style to use for references.  It controls the sort order of the entries in the bibliography and also the formatting for the in-text labels.
%\bibliographystyle{plain}
% This specifies the location of the file containing the bibliographic information.  
% It assumes you're using BibTeX (if not, why not?).
%\bibliography{../../bib/thesis}
% The following statement causes the title "References" to be used for the biliography section:
%\renewcommand{\bibname}{References}
%\addcontentsline{toc}{chapter}{\textbf{References}}
% Tip 5: You can create multiple .bib files to organize your references. 
% Just list them all in the \bibliogaphy command, separated by commas (no spaces).

% The following statement causes the specified references to be added to the bibliography% even if they were not 
% cited in the text. The asterisk is a wildcard that causes all entries in the bibliographic database to be included (optional).
%\nocite{*}
