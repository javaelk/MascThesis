\chapter{introduction}\label{chap:intro}


Regression Test Selection(RTS) is a research area aimed at selecting a subset
of test cases from exsiting test suite for regression testing. Many RTS
techniques have been proposed and evaluated , %1)test selection is effective
empirical evaluation of these technqiues indicats that algorithms can be very
effective in reducing the size of the test suite while still maintain
safty\cite{}. One of the challenges for test organizations adopt RTS techniques
is to determine which specific RTS technique to use. Our proposed STS framework helps test engineers explictly express their test selection goals, systematically analyze software artifacts available in the development process, and select a STS technique provides the most benifits with lowest cost. This framework also helps the research community access to published STS techniques and reproduce the experiments or
conduct new experiments.




2)effectiveness of each technqiues varies
 these studies also show that the facors such as the structure of the
 code,location of the changes, granularity of hte test suite, and frequency of
 the testing affect the reduciton in test suite size that can be achieved by the
 technqiues.

3)how engineers select which techqniues to use? 
Testing is a costly process that can yield over half of total
devopment cost. Regardless the investment on testing, exhaustive testing is not
practical in the industry. In most industry software projects, some forms of
test selection do exist. However, the  Traditionally, decision on which test
selection technqiues to use is made mostly ad hoc. Test engineers may choose a
particular technique based on previous empirical evaluation results of the
technique. However, as empirical evaluation is high dependent on the test
subjects. \cite{DBLP:conf/esem/KasurinenTS10}


On the other hand, it does not make economical sense if the cost of choosing a
good test selection technqiue is higher than the cost of test selection itself.
Or more precisely, the benefit received from test selection should higher
than the cost of the test selection , including cost of choosing a
test selection techqniue and applying the tecnique.
\underline{need to find out an abbreviation of software test selection
technqiues - STS techniques?}


a safe regression test selection technique is one that , under certain
assumptions, selects every test case from the original test suite that can
expose fault in the modified program.\cite{DBLP:journals/tse/RothermelH96}

To our knowledge, this paper is the first to propose a framework to select STS
technqiues. Most of the STS researches to date are focusing on
finding an effective STS technqiue and performing post morterm empirical
evaluation of techqniues. These focus are not surprising from academia's point of
view as the goal of the research is to find an effective techqnique.
However,from engineering point of view, the first challegne is to determine
which STS technqiue to use. 

\chapter{Releated Work}\label{ch:related-work}
%======================================================================
In this chapter, we survey research results related to Test Case Selection
problem. Various approaches have been proposed using different
techniques. Similar to \cite{Yoo:2010}, we surveryed literature has been
published in major software engineering domain, especially in software testing and softwar
maintenance literature. Different from \cite{Yoo:2010}, our focus is on software
test selection. Though some minimization and prioritization literatures also share
similar goals and technqiues as STS. Another difference from \cite{Yoo:2010}is
that we first group the literatures by their goals and required data, then group by
technqiues. This way is more aligned with the determination process in our
framework. The following subsections describe these literatures in details and
highlighting their strengths and weaknesses.

Arafeen and Do\cite{DBLP:conf/issre/ArafeenD11} applied Analytic Hierarchy
Process(AHP) method for selecting test prioritization
techniques.\underline{talk more about how it's done, what is AHP etc}. It's
closely related to our research as it's also trying to address the problem of selecting a technique for test practitioners.The empirical study results show AHP is effective for
selecting appropriate test case prioritization techniques across
system lifetime. AHP outperformed 15 out of 20 cases when the total cost-benefit
values are considered, AHP outperformed 16 out of 20 cases when total number of version that performed best is considered.
However, the pairwise comparison values in AHP is subject to human judgement
and thus the results can be biased by personnel's knowledge and experience.
In the empirical study of \cite{DBLP:conf/issre/ArafeenD11}, values in AHP are
assigned based on history data regarding the performance of test case prioritization techniques observed
from previous empirical studies. However, empirical evidence for differences
between techniques is not very strong, and sometimes contradictory\cite{}. Also
practicality point of view,  assigning values in AHP could be a daunting task
for a test practitioner who do not have to the access or the knowledge of
previous empirical studies in academia. 

%----------------------------------------------------------------------

\section{backgroud}

\section{Code modification based techniques}
%----------------------------------------------------------------------

Many researchers proposed techniques focusing on selecting all nonobsolete tests
in $T$ that execute changed code from $P$ to $P^\prime$. These techniques are
called safe techniques as they select all modification-traversing test cases in
$T$. As $T_{fr} = T_{mr} \le T_{mt}$, these techniques most likely are not
precise, i.e. not all selected modification-traversing test cases will reveal
fault. These techniques typically require knowledge of code coverage information
of each test case $t$ in $T$ and source code of $P$ and $P^\prime$.




%See equation \ref{eqn_pi} on page \pageref{eqn_pi}.\footnote{A famous
%equation.}


\newpage
\section{Cost Models}
Do et al. has proposed and applied comprehensive economic model for
regression testing process
~\cite{DBLP:conf/issta/DoR08,DBLP:conf/sigsoft/DoMTR08,DBLP:conf/sigsoft/DoR06,5482587}.
This cost model captures both cost and benefit factors relavant to regression testing process. The cost model is used to support evaluation of various
regression test selection techniques.
\\The major difference between our cost
model and cost model presented in \cite{DBLP:conf/issta/DoR08} is that their
cost model is post mortem, i.e the cost model is to evaluate the effect of
regression testing techniques after the regression testing phase completed.
Our cost model is to provide cost estimation prior to apply any test selection technique. As suggested in \cite{DBLP:conf/sigsoft/DoR06}, test selection technique might vary with test suite type. Providing cost estimation to practioners up front will help them to make consicous decision on which test selection technique to choose for a specific release or test suite.
Our cost model is based on \cite{DBLP:conf/issta/DoR08} and has removed factors
can not be abtained or resonablly estimated before regression testing
start.Specificially, we made following changes to Do's cost
model.\begin{itemize}\item Remove Delayed fault detection feedback(CD).CD
captures the cost of delaying product release due to late discovered faults. CD directly concerns with order of the test cases.Cost of delayed fault feedback would be low if test cases are
executed in a priority order such that faults are found early in the regression
testing cycle and sufficent time is given to fix the faults prior to
release date. However, our cost model is to provide cost estimation on
various test selection techniques. We do not concerns with the order of test
cases therefore we removed this cost factor from our model.\item Do's cost
model accounts for lifetime factors by tracking costs and benefits across entire
sequences of system releases. However, it's very difficult to estimate the
number of release $n$ prior to regression testing phase. There are research in
predicting total number of faults in a
release\cite{DBLP:journals/smr/RajaHH09}. However, the number of releases
built in regression testing phase may not directly related to total number of faults corrected. Some organizations create new bulid on a time boxed schedule, e.g one build per week; some organizations create new build on demand, e.g. only when there are significant faults are corrected a new build will
be created. In our experiement, we assume a fixed daily build. \item Consider
the cost of Known Fault.\underline{Explain different test execution status
passed,failed, blocked}. When there are difference resources to fix faults and
\end{itemize}

\section {A surver of software test optimization goals}
\underline{a comprehensive list of all papers and what are the goals,organize
based on goals and subgoals,incl. selelction,prioritzation and reduction}. 
Many software test optimization techniques including test selection,
prioritization and reductin technqiues have been researched over the
years.Researchers normally have one or many goals implcitly or explictly in mind
when researching techqniuqes. In this section, we survery the goals behind
current software test optimizatioin researchs to date and point out how these
goals impact our framework design.


\section {Goal Modelling}
 %3 ways to model goals- AND-OR graph,SIG, Lattice , what is goal modelling in
 % general. what is SIG, what others use and why we don't use it.
Goal Modelling is an important element in requirments engineering to visulize
functional or non-funcitonal requirements. It's also used widely in Process Analysis, Business Analysis and other engineering domains. A goal in Requirement Engineering is a stakeholder's objective for the system. A goal relates to one or many low-level requirements. A goal
model is a set of nodes representing hierarchically decomposed goals plus a
set of edges representing both decomposition and contribution relationships
among goals\cite{Ernst:2006:VNR:1173697.1174072}.Goal modeling helps engineers systematically formalize stakeholder goals and visulaze hireacy and relationship of goals.

Goal models can also be used to capture actors and dependencies
\cite{Ernst:2006:VNR:1173697.1174072,DBLP:conf/icse/YuM94} An Actor Dependecny
model consists of a set of nodes to represent actors , and each link between two
actors indicates that one actor depends on the other in order that
the former may attain some goals.
 
Salehie et al. use GQM approach to identify goals and  metrics for test case
priroitziation. GQM does not capature interdependcey between goals. 
 
Soft goals are goals can be partially achieved.


%May 30,2011
Chung et al. proposed a Softgoal Interdependecy Graphs in \cite{NFR} as part of
the non-functional requirement framework. The major concepts of the framework is
a graph representation of softgoals and interdependices among softgoals.
In SIG, Softgoals, which are ``soft'' in nature, are shown as clouds. Softgoals have associated labels (values representing the degree to
which a softgoal is achieved) which are used to support the reasoning process
during design. Softgoals can also have an associated prioirty
(criticality or dominace). Softgoals can also have other attributes, such as
author, and time of creation of the softgoal.
Softgoals are connected by interdependecy links, which are shown as lines, often
with arrowheads. Interdependencies show refinements of ``parent'' softgoals downwards into other, more specific, ``offspring'' softgoals. They also show the contribution (impact) of offspring upwards upon the meeting of other (parent) softgoals.

In the following, we provide more details softgoal interdependancies about SIG
based on \cite{NFR}.



In the context of this paper, we use goal models to capture
stakeholders motiviations and intentions to perform test selection. A goal is a
stakeholer's objective for performing test case selection. Stakeholder typically are test engineers or test managers who wish to perform test selection with some objectives in mind. A goal relates to one or
many test selection techniques. 
Engineers select test cases with some test selection technqiues need to consider
the needs and wants from other stakeholders in the organization. Goal modeling
can help engineers to have deeper understanding of various goals from other
software engineers, managers and customers in order to choose the best test
selection techniques and ultimately deliver the best product balance all
stakeholders needs.











\chapter{Introduction}
We expect this framework to help test engineers make better choice of STS
techniques that are most apporiate for their organization and processes.
Challeges- empirical results various for different program.

% ======================================================================
\chapter{3}
% ======================================================================


\section{ Constraint satisfaction problem}
%\href{http://en.wikipedia.org/wiki/Constraint_satisfaction_problem}{Constraint
%satisfaction problem}

\section{Design of Experiments}
In the design of experiments, an idependent variable's values are controlled or
selected by the experimenter to determine its relationship to an observed
phenomenon (i.e. the dependent variable). 

In summary:
Independent variables answer the question "What do I change?"
Dependent variables answer the question "What do I observe?"
Controlled variables answer the question "What do I keep the same?"
Extraneous variables answer the question "What uninteresting variables might mediate the effect of the IV on the DV?"
This would be a good place for some figures and tables.
