% UW Thesis Template for LaTeX 
% Last Updated July 8, 2009 by Stephen Carr, IST Client Services
% FOR ASSISTANCE, please send mail to rt-IST-CSmathsci@ist.uwaterloo.ca

% Effective October 2006, the University of Waterloo 
% requires electronic thesis submission. See the UW thesis regulations at
% http://www.grad.uwaterloo.ca/Thesis_Regs/thesistofc.asp.
% However, many faculties/departments also require one or more printed
% copies. This template attempts to satisfy requirements for both types of output. 
% It is based on the standard "report" document class. The "book" document 
% class can be substituted if you have a very large, multi-part thesis.

% DISCLAIMER
% To the best of our knowledge, this template satisfies the current UW requirements.
% However, it is your responsibility to assure that you have met all 
% requirements of the University and your particular department.
% Many thanks to Colin Alie for assistance in preparing this updated template.

% -----------------------------------------------------------------------

% By default, output is produced that is geared toward generating a PDF 
% version optimized for viewing on an electronic display, including 
% hyperlinks within the PDF.
 
% E.g. to process a thesis called "mythesis.tex" based on this template, run:

% pdflatex mythesis	-- first pass of the pdflatex processor
% bibtex mythesis	-- generates bibliography from .bib data file(s) 
% pdflatex mythesis	-- fixes cross-references, bibliographic references, etc
% pdflatex mythesis	-- fixes cross-references, bibliographic references, etc

% If you use the recommended LaTeX editor, Texmaker, you would open the mythesis.tex
% file, then click the pdflatex button. Then run BibTeX (under the Tools menu).
% Then click the pdflatex button two more times. If you have an index as well,
% you'll need to run MakeIndex from the Tools menu as well, before running pdflatex
% the last two times.

% N.B. The "pdftex" program allows graphics in the following formats to be
% included with the "\includegraphics" command: PNG, PDF, JPEG, TIFF
% Tip 1: Generate your figures and photos in the size you want them to appear
% in your thesis, rather than scaling them with \includegraphics options.
% Tip 2: Any drawings you do should be in scalable vector graphic formats:
% SVG, PNG, WMF, EPS and then converted to PNG or PDF, so they are scalable in
% the final PDF as well.
% Tip 3: Photographs should be cropped and compressed so as not to be too large.

% To create a PDF output that is optimized for double-sided printing: 
%
% 1) comment-out the \documentclass statement in the preamble below, and
% un-comment the second \documentclass line.
%
% 2) change the value assigned below to the boolean variable
% "ElectronicVersion" from "true" to "false".

% --------------------- Start of Document Preamble -----------------------

% Specify the document class, default style attributes, and page dimensions
% For hyperlinked PDF, suitable for viewing on a computer, use this:
%----------------------------------------------------------------------
% MAIN BODY
%----------------------------------------------------------------------
% Because this is a short document, and to reduce the number of files
% needed for this template, the chapters are not separate
% documents as suggested above, but you get the idea. If they were
% separate documents, they would each start with the \chapter command, i.e, 
% do not contain \documentclass or \begin{document} and \end{document} commands.
%======================================================================
\reversemarginpar
\setlength{\marginparwidth}{2cm}
\chapter{Proposing Selection Techniques}
\label{chap:Proposing Selection Techniques}
%======================================================================
\section{Precision Predictor}\label{sec:Precision Predictor}
\underline{talk about why do we need precision predictor, rw, and others}

Precision] measures the extent to which a technique select
l  test cases that are non-modification-revealing.

\section{Cost Predictor}
\label{sec:Cost Predictor}
\underline{focus on how to predict analysis cost, rather than why need to make
the prediction}

\underline{an updated technique modelling diagram showing implementation has
cost and precision predictor}
%----------------------------------------------------------------------
\section{Techniques Determination}
\label{sec:Techniques Determination}
In section~\ref{sec:Precision Predictor}, we discussed Precision
Predictor component in our Framework. The Precision Predictor component provides
valuable precision prediction for researchers or test engineers to compare one
technique over another.  However, it's likely that precision is not the only metric to
determine a test selection technique as technique with lower precision value
(ie. less percentage of test cases selected from original test suite) is not
always a better technique. To identify what is a better technique, initially
researchers have proposed analytical approaches to access several attributes
of techniques and compare techniques\cite{DBLP:journals/tse/RothermelH96}.
\begin{description}
\item[Precision] measures the extent to which a technique select
test cases that are non-modification-revealing.
\item[Inclusiveness] measures the extent to which a technique
omits tests in $T$ that reveal faults in a modified program. a 100 percent
inclusive technique is safe.
\item[Efficiency] measures the space and time requirements of a
technique.
\item[Generality] measures the ability of a technique to function in a practical
and sufficiently wide range of situations.
\end{description} 

\underline{Q:These factors are good to consider when evaluate STS techniques
post-mortem. Can we actually use them prior to regression test? Can we use
evaluation results from previous experiments/history values?}%
These attributes help to compare the strength and weakness of techniques in
general, however it's a qualitative analysis rather than quantitative. Our work
goes beyond goal-based STS techniques by allowing researchers and test
engineers to use a utility function to determine which technique to use for the
application under test. Our utility function incorporate the attributes
mentioned about and produce one single utility value.

Utility is a value that represents the desirability of a particular state or
outcome.\underline{need a better reference to utility here}
%\href{http://www.pdl.cmu.edu/PDL-FTP/Storage/CMU-PDL-08-102.pdf}{utility function section 1.1.3}
The utility function provides the ability to collapse
multiple objectives into a a single axis that can be used to compare with all
candidate STS techniques.Using utility, both the cost and benefits of each STS
technique can be examined in a single framework. Using utility function also
allows an automated optimization system to properly balance cost and benefits of each STS
techniques.

Ultimate goal of any selection is to find a technique gives the highest
benefits with lowest cost. Our utility function predicts cost-benefit on each candidate
techniques based on the output from Precision Predictor and Cost Predictor. Our
utility function is based on cost model presented in 
\cite{DBLP:conf/sigsoft/DoR06} but it's different from it in  
\begin{itemize}
  \item Our Utility Function is applied prior to test selection and test
  execution phase, while D-R cost model is applied after test
  execution phase for post-mortem analysis
  \item Utility value calculated from Utility Function does not represent cost
  in dollar amount. The utility value does not have a measurement unit. It's
  just a single value to compare one technique to another. D-R cost model does
  represent cost in dollars, therefore it requires additional parameters like
  organization's revenue in dollars per time unit, average hourly salary
  associated with employing a programmer per unit of time. Additional cost
  parameters in D-R cost model requires estimation inputs from
  each organization/teams and makes it more complex then our Utility Function.
  \item Cost components constant to any STS techniques are not included in our
  Utility Function as they do not contribute any difference in comparing STS
  techniques. This further simplifies our Utility Function. 
\end{itemize} 

\underline{describe cost model}
\subsection{Cost Components}\label{sec:cost_components}
In the following section, we describe cost components proposed in
D-R cost model\cite{DBLP:conf/sigsoft/DoR06} with emphasis on
which components are included in the Utility Function and how these components
are included. A summary of all cost components in relation to our Utility
Function is listed in table \ref{}. Methods for measuring or estimating these
components in our framework implementation are described in more details in
section~ref{chap:Empirical Evaluations}.
\begin{description}
\item[Test Setup]($CS$). $CS$ includes cost of activities
required to prepare to run tests.This includes costs of setting up test environments,
configuring test drivers and test simulators. This cost is necessary to execute
the test suite regardless if entire test suite is selected or only one test
case is selected. $CS$ is not dependent on selection technique,
therefore it's not included in the Utility Function.
\item[Identifying obsolete test case]($CO_i$). $CO_i$ represents
cost to determine which test cases are still applicable to a new version of the
program. Similar to $CS$, this cost is necessary to execute
the test suite regardless if entire test suite is selected or only one test
case is selected. $CO_i$ is not dependent on selection technique,
therefore it's not included in the Utility Function.
\item[Repairing obsolete test cases]($CO_r$). $CO_r$ represents
cost to repair obsolete test cases for reuse.$CO_r$ is necessary to
execute the test suite regardless if entire test suite is selected or only one
test case is selected. $CO_r$ is not dependent on selection technique,
therefore it's not included in the Utility Function.
\item[Supporting analysis]($CA$). $CA$ represents the cost of
analysis needed to support a STS technique. $CA$ includes cost of
instrumenting code, analyzing changes and collecting test execution traces. In
our Utility Function, this is the predicated analysis cost. As described in
section ref{sec:Cost Predictor}, each unique STS technique in the framework
implements a Cost Predictor to predicate $CA$ value.
\item[Regression testing technique execution]($CR$). $CR$ represents the cost of
applying a STS technique after analysis has been completed. Each unique STS technique in the framework implements a Cost
Predictor to predicate $CR$ value.
\item[Test execution]($CE$). $CE$ represents the cost of executing
all selected test cases. this includes human and machine costs for all manual,
automatic , or semi-automatic test cases. $CE$ is dependent on the number
of test cases selected for regression. If the run time of each test case is
uniformed in a test suite, $CE$ cost is less when less number of
test cases is selected. In our Utility Function, we make simplifying assumption
that each test in the test suite has equal run time. We use $CE_avg$
denotes average execution cost of each test case. 
\item[Test result validation]($CV_d and CV_i$). $CV_d and CV_i$ represent the
cost of checking test results to determine whether or not test case reveal failures. $CV_d$ is the cost of comparing test results
automatically, $CV_i$ is the cost of comparing test results
manually. $CV_d and CV_i$ are dependent on the number of test cases
selected. In our Utility Function, we make simplifying assumption that each test
in the test suite has equal test result validation time. We use $CV_avg$
denotes average test result validation time for each test. $CV_avg$
includes both manual and automated result validation time.
\item[Missing faults]($CF$). $CF$ represents the cost of missing
faults. This includes the costs of missing fault that the test suite could have
detected if executed in full, but missed due to omission of some test cases. If
the STS technique is a safe technique, we do not count $CF$ in the
Utility Function. \underline{really?what is a safe technique}
\item[Delayed fault detection feedback]($CD$). $CD$ is the cost
of delaying product release due to late discovered faults. As a simple example,
if bug fixing cycle is two days but a fault is found one day before scheduled
release time, the release has to be delayed for at least one day in order to
take a new build with the fix of that late discovered fault. This cost could occur
when fault revealing test cases were executed too late in the testing cycle and
high $CD$ cost indicates a test case prioritization issue. Clearly,
$CD$ is dependent on execution orders of the test cases but not dependent
on STS techniques. Test cases selected by the same STS technique could be
executed in an optimized order to incur zero or low $CD$ cost or in an
un-optimized order to incur high $CD$ cost. Therefore, the Utility
Function in our framework do not consider $CD$ cost component.
\end{description}
\underline{discuss which cost in which phase} Recall in section~ref{sec:???}, we
discussed testing phases. 

%% Table generated by Excel2LaTeX from sheet 'TABLE FOR LATEX'
\begin{table}[htbp]
\begin{small}
  \centering
  \caption{Summary of Cost Components in D-R Cost Model and Utility
 Function}
    \begin{tabular}{rrrrrrr}
   % \toprule
    \textbf{Cost} &    & \textbf{Included}
    & \textbf{Included} & \textbf{Technique} &
     & \textbf{Human} \\
    \textbf{Component} & \textbf{Description} & \textbf{in D-R}
    & \textbf{in Utility} & \textbf{dependent} &
    \textbf{Phase} & \textbf{/machine cost} \\
    
    %\midrule
    $CS$ & Test setup & Y & N & N &Pre-regression & Human \\
    $CO_i$ & Identifying obsolete test cases & Y     & N     & N     &
    Pre-regression & Human \\
    $CO_r$ & Repairing obsolete test cases  - & Y     & N     & N     &
    Pre-regression & Human \\
    $CA_{in}$ & Instrumenting all units & Y     & Y     & Y     & Pre-regression
    & Machine \\
    $CA_{tr}$ & Collecting traces for test cases  & Y     & Y   & Y     &
    Pre-regression& Machine \\
    $CR$ & Regression testing technique execution & Y     & Y     & Y    &  Pre-regression & Machine \\
    $CE$ & Test execution & Y     & N     & Y     &   Regression & Human\&Machine \\
    $CE_{avg}$& Test execution per test case & N     & Y     & N   & Regression & Human\&Machine \\
    $CV_d$ and $CV_i$ & Test result validation& Y     & N     & Y     & Regression & Human\&Machine \\
    $CV_{avg}$ & Test result validation per test case & N     & Y     & N     & Regression& Human\&Machine \\
    $CF$ & Missing faults & Y     & Y     & Y     & Post - release  &    Human\&Machine \\
    $CD$ & Delayed fault detection feedback& Y     & N     & N     & Regression & Human\&Machine \\
    %\bottomrule
    \end{tabular}%

\end{small}
\label{tab:Summary_of_Cost_Components} 
\end{table}%


Table~\ref{tab:Summary_of_Cost_Components} summarizes cost components in D-R
cost model and cost components considered in our Utility Function.
\underline{describe more about the table for each column and abbreviation if
not clear}


\pagebreak

\subsection{Utility Function}\label{Utility_function}
The utility function is ultimately a simplified cost model to perform cost
benefit analysis prior to test selection and after test selection. Before we
discuss Utility Function, we define several parameters and coefficients that are
used in the Utility Function, most of which instantiated cost components outlined in section~\ref{sec:cost_components}. Assume we have a candidate list of STS techniques $M_1$,$M_2$,\ldots,$M_k$, $n$ versions of
programs $P$ and $n$ versions of test suite $T$ (one per version of program P).
Set of test cases selected by technique $M$ denoted $T_m$.
\begin{itemize}
  \item $i$ is an index denoting a particular version $P_i$ of $P$.
  \item $k$ is an index denoting a particular technique $M_k$ of $M$.
  \item $CA_{in}(i)$ is the cost of instrument all units in $P_i$.
  \item $CA_{tr}(i)$ is the cost of collecting traces for test cases in
  $T_{i-1}$ over $P_{i-1}$ 
  \item $CR(i)$ is the cost of applying test selection technique $M_i$
  \item $CE_{avg}(i)$ is the average execution cost of one test case in $T_i$
  \item $CV_{avg}(i)$ is the average test result validation cost of one test
  case, including automated and manual tests
  \item $CF$ is the cost of missing faults that the regression test suite
  could have detected if executed in full.
  \item $\theta_c$ ratio of cost in regression phase(critical phase) over
  pre-regression phase (preliminary phase)
  \item $\theta_p$ ratio of cost in post-release phase over pre-regression phase
  (preliminary phase)
   
\end{itemize}

The two equations that comprise our Utility Function are as follows: 
$$ Cost = \sum_{i=2}^{n}(CA_{in}(i)+CA_{tr}(i)+CR(i)+CF(i)*\theta_p)$$ 
$$ Benefit = \sum_{i=2}^n((|T(i)|-|Tm(i)|)*(CE_{avg}(i)+CV_{avg}(i))*\theta_c)$$

Utility Function $$ \digamma = Benefit - Cost $$

The Utility Function is applied  twice in the framework,once in Proposing Selection Technique phase, once in EValation
phase. In the Proposing Selection Technique phase, the Utility Function is
applied to all candidate STS techniques and utility values are calculated.
Technique with the highest utility value is proposed. In Evaluation phase,  the
Utility Function is once again applied to all candidate STS techniques and
utility values are re-calculated. The proposed technique is evaluated against
technique with the highest actual utility value.

One of the assumptions we made in the utility function above is that
every test case selected will always be executed. This may not be true in some
situations especially in organizations with separate test teams and development
teams.A new version of the program could be kicked off by development team while test team has not yet finished execution of the
selected test set $T_m$ on current version. In this case, test team has
several choices based on organization's processes and the utility function can
be applied with some modifications discussed below.
\begin{itemize}
  \item continue to execute the remaining test cases on the current
  version and ignore the new version until all tests are completed. In
  this case, the utility function can be used without any modification.
  \item stop test execution on current version. Reselect test cases based on the
  new version and execute the newly selected test cases on new version. In this
  case,test team essentially performed a unsafe test selection on current
  version. All test cases executed so far against the current version
  are selected;all remaining test cases in the test suite are considered
  discarded. We can adjust the cost component in the utility function so that
  $T_m$ is the set of test cases executed on current version;$CF$ is the
  cost of missing faults due to omitting remaining test cases.
  \item execute the remaining test cases on the new version without reapplying
  test selection technique. In this case, $n$ no longer represents number of
  program versions as test selection is not perform on each every program
  version. Instead, $n$ represents number of times test selection was performed
  in a release. $i$ becomes an index denoting a particular test selection performed over a program version. 
\end{itemize}


\subsection{estimation of utility parameters}
In this section, we discuss in more details how to estimate/measure each
parameters in the Utility Function and challenges around estimation and
measurement in practice. As the utility function is applied twice in the
framework , once in Proposing Selection Technique phase, once in Evaluation
phase, some parameters in the utility function are estimated/measured twice as
well. 

\begin{itemize}
  \item $i$ is an index denoting a particular version $P_i$ of $P$. This seems
  to be always known in any research experiments but this may not be known in
  practice at technique determination time. 
  \item $k$ is an index denoting a particular technique $M_k$ of $M$. This is
  the number of technique implemented in the framework and this is a known fact.
  \item $CA_{in}(i)$ is the cost of instrument all units in $P_i$.
  \item $CA_{tr}(i)$ is the cost of collecting traces for test cases in
  $T_{i-1}$ over $P_{i-1}$ 
  \item $CR(i)$ is the cost of applying test selection technique $M_i$
  \item $CE_{avg}(i)$ is the average execution cost of one test case in $T_i$
  \item $CV_{avg}(i)$ is the average test result validation cost of one test
  case, including automated and manual tests
  \item $CF$ is the cost of missing faults that the regression test suite
  could have detected if executed in full.
  \item $\theta_c$ ratio of cost in regression phase(critical phase) over
  pre-regression phase (preliminary phase)
  \item $\theta_p$ ratio of cost in post-release phase over pre-regression phase
  (preliminary phase)
\end{itemize}


\subsection{When do you want to disable cost model in our framework}
In previous sections, we describe a cost model which is the base of the utility
function in our framework and provided some guidelines about estimation and measurement of the parameters
in the utility function. Later in chapter \ref{chap:Empirical Evaluations}, we
will discuss values of these parameters used in our empirical evaluations.
However, for test practitioners to use the framework in the industry, we
propose disable the utility function in the framework. In the
following section, we will elaborate our reasons.

In practice, even a simplified cost model as we described in
section~\ref{Utility_function} can not be put into use in a single step. It must
be refined and improved iteratively over time. For example, the COCOMO
model \cite{boehm2000software} has gone through several calibration and
evaluation steps, and it took 15 years over 63 projects to improve model's
accuracy. Years of calibration and evaluation is a significant investment for
any organization. Cost model calibration requires estimating the parameters and applying the model at technique determination phase, then tracking and calculating the actual values at evaluation phase for every projects within the organization. 
This would be a significant overhead to the product development and testing and
it could be the major road block in integrating software test selection into testing
processes.From overall cost-effective point of view, there may not be any
benefit until an effective cost model is built.
Also as software development practice and software product evolves so rapidly in the industry, history data collected in previous
project may become obsolete in just number of years to even a few months. Few
companies would be able to afford 15 years to fine tune a cost model.In the end, the return of building an accurate cost model
to determine the ``best'' test selection technique may not even cover the cost
of years of calibration and evaluation. Confronting between dilemma of building
an accurate cost model to determine the ``best'' technique or estimating a
``good'' technique to apply to right away, many organization in the industry
would choose the latter. 

In general, building the utility function for an organization requires 
expert knowledge in cost modelling, manual efforts to collect and enter
parameters, and significant cost to get it right. Since one of the major challenges for industry to adopt test selection techniques is lack of
user friendly tools\cite{4659253}, we suggest organizations disabling
the utility function in the framework, so that the framework becomes an automated and easy to use tool with
low setup cost and minimum manual intervention. After organizations pass the
initial hurdles and adopt the automated test selection practice, they may enable
the utility function to get greater benefits from test selection techniques.

 
Organizations may decided not to fix a bug in a particular release (i.e add the bug into a known issue list) but it makes no sense not to discover a bug in
testing phase. therefore, from practically point of view, test teams would
lalways favour on safe techniques over unsafe technique. 


  
%----------------------------------------------------------------------
% END MATERIAL
%----------------------------------------------------------------------
% B I B L I O G R A P H Y
% -----------------------

% The following statement selects the style to use for references.  It controls the sort order of the entries in the bibliography and also the formatting for the in-text labels.
%\bibliographystyle{plain}
% This specifies the location of the file containing the bibliographic information.  
% It assumes you're using BibTeX (if not, why not?).
%\bibliography{../../bib/thesis}
% The following statement causes the title "References" to be used for the biliography section:
%\renewcommand{\bibname}{References}
%\addcontentsline{toc}{chapter}{\textbf{References}}
% Tip 5: You can create multiple .bib files to organize your references. 
% Just list them all in the \bibliogaphy command, separated by commas (no spaces).

% The following statement causes the specified references to be added to the bibliography% even if they were not 
% cited in the text. The asterisk is a wildcard that causes all entries in the bibliographic database to be included (optional).
%\nocite{*}



