%-------------------------------------------------------------------------------------------------------------
% DOCUMENT STYLE
%-------------------------------------------------------------------------------------------------------------
\documentclass[11pt,dvips]{library/thesis91e}
%-------------------------------------------------------------------------------------------------------------


%-------------------------------------------------------------------------------------------------------------
% SPACING
%-------------------------------------------------------------------------------------------------------------
\usepackage{setspace}        % for use of \singlespacing and \doublespacing
%-------------------------------------------------------------------------------------------------------------


%-------------------------------------------------------------------------------------------------------------
% VERBATIM
%-------------------------------------------------------------------------------------------------------------
\usepackage{moreverb}        % Using this package to get better control of the
                             % verbatim environment, mostly for the use of the
                             % listing environment which puts line number
                             % beside each line.  Note that there has to be a number
                             % in each set of brackets, i.e., \begin{listing}[1]{1}.
                             % PDF info file is "The moreverb package" by
                             % Robin Fairbairns (rf@cl.cam.ac.uk) after
                             % Angus Duggan, Rainer Schopf and Victor Eijkhout, 2000/06/29.
%-------------------------------------------------------------------------------------------------------------
%\usepackage{verbatim}        % allows the use of \begin{comment} and \end{comment}
                             % as well as \verbatiminput{file}
                             % Note:  when using verbatim to input from a text file,
                             % such as a specification or code, use \begin{singlespacing}
                             % and \end{singlespacing}.  Also, tabs are not read
                             % properly, so the input file must only use spaces.

%                             \begin{comment}
%                             Can also use the verbatim package for
%                             comments like this...
%                             \end{comment}
%-------------------------------------------------------------------------------------------------------------


%-------------------------------------------------------------------------------------------------------------
% MATH STUFF
%-------------------------------------------------------------------------------------------------------------
\usepackage{amsmath}         % to make nice equations

%-------------------------------------------------------------------------------------------------------------
\usepackage{amsthm}          % to make nice theorem, i.e., definition

% Using the amsthm package, define a new theorem environment for my
% definition.  * means don't number it.
%\newtheorem*{definition}{Definition}
%-------------------------------------------------------------------------------------------------------------
\usepackage{cases}           % to make numbered cases (equations)
%-------------------------------------------------------------------------------------------------------------
\usepackage{calc}            % Used with the Ventry environment defined below.
%-------------------------------------------------------------------------------------------------------------

%-------------------------------------------------------------------------------------------------------------
% FLOATS AND FIGURES
%-------------------------------------------------------------------------------------------------------------
\usepackage{graphicx}        % for graphic images (use \includegraphics[...]{file.eps})
%-------------------------------------------------------------------------------------------------------------
\usepackage{subfigure}       % for subfigures (figures within figures)
%-------------------------------------------------------------------------------------------------------------
\usepackage{boxedminipage}   % to make boxed minipages, i.e., boxes around figures
%-------------------------------------------------------------------------------------------------------------
\usepackage{rotate}          % for use of \begin{sideways} and \end{sideways}
%-------------------------------------------------------------------------------------------------------------
\usepackage{float}           % Using this package to get better control of my floats
                             % including the ability to define new float types for
                             % my specification and code listings.
                             % DVI info file is "An Improved Environment for Floats"
                             % by Anselm Lingnau, lingnau@tm.informatik.uni=frankfurt.de
                             % 1995/03/29.

% Define new float styles here
% Ruled style for examples
%\floatstyle{ruled}
%\newfloat{Example}{h}{lop}[chapter]

% Style of float used for code listings
\floatstyle{ruled}
\newfloat{Listing}{H}{lis}[chapter]

                             % Note:  The listings don't have space between the chapters, unlike
                             % the standard list of tables etc.  At the end, copy the spacing
                             % commands from the .toc file and insert into the .lis file.  Then,
                             % DO NOT LATEX it again, simply go to the DVI viewer!
%-------------------------------------------------------------------------------------------------------------
% TABLES
%-------------------------------------------------------------------------------------------------------------
\usepackage{tabularx}        % Package used to make variable width-columns, i.e.,
                             % column widths are changed to fit the maximum width
                             % and text is wrapped nicely.

\usepackage{threeparttable}
%-------------------------------------------------------------------------------------------------------------
% CAPTIONS
%-------------------------------------------------------------------------------------------------------------
\usepackage[hang]{caption}   % Package used to make my captions 'hang', i.e., wrap
                             % around, but not under the name of the caption.
%-------------------------------------------------------------------------------------------------------------
% Find that the captions are too far from my verbatim figures, but if
% I change it to 0, then the captions are too close for my other types
% of figures.  Maybe set each one separately?
%\setlength{\abovecaptionskip}{1ex}

%\setlength{\textfloatsep}{1ex plus1pt minus1pt}

%\setlength{\intextsep}{1ex plus1pt minus1pt}

%\setlength{\floatsep}{1ex plus1pt minus1pt}
%-------------------------------------------------------------------------------------------------------------


%-------------------------------------------------------------------------------------------------------------
% MISCELLANEOUS
%-------------------------------------------------------------------------------------------------------------
\usepackage{layout}          % useful for determining the margins of a document
                             % use with \layout command
%-------------------------------------------------------------------------------------------------------------
\usepackage{changebar}       % Way of indicating modifications by putting bars in the
                             % margin.  Read about it in "The Latex Companion".
%-------------------------------------------------------------------------------------------------------------

\usepackage{rotating,psfig,epsfig,lscape}
\usepackage[numbers,sort&compress]{natbib}

%-------------------------------------------------------------------------------------------------------------


%-------------------------------------------------------------------------------------------------------------
% MISCELLANEOUS COMMANDS AND ENVIRONMENTS
%-------------------------------------------------------------------------------------------------------------

%-------------------------------------------------------------------------------------------------------------
% MY DEFINED COMMANDS
%-------------------------------------------------------------------------------------------------------------

% Command that I can use to create notes in the margins;
% adapted from Juergen's META tag
\newcommand{\meta}[1]{\begin{singlespacing}
{\marginpar{\emph{\footnotesize Note: #1}}}\end{singlespacing}}

% Command that I can use to create lined headings
\newcommand{\heading}[1]{\bigskip \hrule \smallskip \noindent \texttt{#1} \smallskip \hrule}

% Command that I can use for reading in a file, verbatim, with line
% numbers printed along the left side.  The parameter is the file name.
\newcommand{\fileinnum}[1]{
    \begin{singlespacing} {\footnotesize
    \begin{listinginput}[1]{1}{#1}\end{listinginput}
    }\end{singlespacing}
}


% Command that I can use for reading in a file, verbatim, with NO line
% numbers, but in a smaller font.  The parameter is the file name.
\newcommand{\filein}[1]{
    \begin{singlespacing}{\footnotesize
    \begin{verbatiminput}{#1}\end{verbatiminput}
    }\end{singlespacing}
}


% Command that I can use for reading in a file, verbatim, with NO line
% numbers, but in a smaller font.  The parameter is the file name.
\newcommand{\fileinsmall}[1]{
    \begin{singlespacing}{\scriptsize
    \begin{verbatiminput}{#1}\end{verbatiminput}
    }\end{singlespacing}
}


% Dean't 'notesbox' command.  Needs setspace package.
%   Usage: \notesbox{This is a note.}
%
\newcommand{\notesbox}[1]{
%     \ \\
      \singlespacing
      \noindent\begin{boxedminipage}[h]{\textwidth}{\sf{#1}}\end{boxedminipage}
      \doublespacing
}

\usepackage{times}
\usepackage{epsfig}
\usepackage{psfig}
\usepackage{graphicx}
\usepackage[all]{xy}
%\usepackage{color}
\usepackage[usenames]{color}
\usepackage{colortbl}
\usepackage{amsmath}
\usepackage{amssymb}
\usepackage{listings}
\usepackage{ifthen}
\usepackage{todonotes}

\usepackage[algochapter,ruled,lined,linesnumbered]{algorithm2e}
\setlength{\algomargin}{4mm}

\definecolor{table-header-color}{rgb}{0.7,0.7,0.7}


\newtheorem{definition}{Definition}[chapter]
\newtheorem{lemma}{Lemma}[chapter]
\newtheorem{theorem}{Theorem}[chapter]
\newtheorem{property}{Property}[chapter]
\newtheorem{heuristic}{Heuristic}[chapter]
\newtheorem{criterion}{Termination Criterion}[chapter]