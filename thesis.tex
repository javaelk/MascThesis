% uWaterloo Thesis Template for LaTeX 
% Last Updated May 24, 2011 by Stephen Carr, IST Client Services
% FOR ASSISTANCE, please send mail to rt-IST-CSmathsci@ist.uwaterloo.ca

% Effective October 2006, the University of Waterloo 
% requires electronic thesis submission. See the uWaterloo thesis regulations at
% http://www.grad.uwaterloo.ca/Thesis_Regs/thesistofc.asp.

% DON'T FORGET TO ADD YOUR OWN NAME AND TITLE in the "hyperref" package
% configuration below. THIS INFORMATION GETS EMBEDDED IN THE PDF FINAL PDF DOCUMENT.
% You can view the information if you view Properties of the PDF document.

% Many faculties/departments also require one or more printed
% copies. This template attempts to satisfy both types of output. 
% It is based on the standard "book" document class which provides all necessary 
% sectioning structures and allows multi-part theses.

% DISCLAIMER
% To the best of our knowledge, this template satisfies the current uWaterloo requirements.
% However, it is your responsibility to assure that you have met all 
% requirements of the University and your particular department.
% Many thanks to the feedback from many graduates that assisted the development of this template.

% -----------------------------------------------------------------------

% By default, output is produced that is geared toward generating a PDF 
% version optimized for viewing on an electronic display, including 
% hyperlinks within the PDF.
 
% E.g. to process a thesis called "mythesis.tex" based on this template, run:

% pdflatex mythesis	-- first pass of the pdflatex processor
% bibtex mythesis	-- generates bibliography from .bib data file(s) 
% pdflatex mythesis	-- fixes cross-references, bibliographic references, etc
% pdflatex mythesis	-- fixes cross-references, bibliographic references, etc

% If you use the recommended LaTeX editor, Texmaker, you would open the mythesis.tex
% file, then click the pdflatex button. Then run BibTeX (under the Tools menu).
% Then click the pdflatex button two more times. If you have an index as well,
% you'll need to run MakeIndex from the Tools menu as well, before running pdflatex
% the last two times.

% N.B. The "pdftex" program allows graphics in the following formats to be
% included with the "\includegraphics" command: PNG, PDF, JPEG, TIFF
% Tip 1: Generate your figures and photos in the size you want them to appear
% in your thesis, rather than scaling them with \includegraphics options.
% Tip 2: Any drawings you do should be in scalable vector graphic formats:
% SVG, PNG, WMF, EPS and then converted to PNG or PDF, so they are scalable in
% the final PDF as well.
% Tip 3: Photographs should be cropped and compressed so as not to be too large.

% To create a PDF output that is optimized for double-sided printing: 
%
% 1) comment-out the \documentclass statement in the preamble below, and
% un-comment the second \documentclass line.
%
% 2) change the value assigned below to the boolean variable
% "PrintVersion" from "false" to "true".

% --------------------- Start of Document Preamble -----------------------

% Specify the document class, default style attributes, and page dimensions
% For hyperlinked PDF, suitable for viewing on a computer, use this:
\documentclass[letterpaper,12pt,titlepage,oneside,final]{book}
 
% For PDF, suitable for double-sided printing, change the PrintVersion variable below
% to "true" and use this \documentclass line instead of the one above:
%\documentclass[letterpaper,12pt,titlepage,openright,twoside,final]{book}

% Some LaTeX commands I define for my own nomenclature.
% If you have to, it's better to change nomenclature once here than in a 
% million places throughout your thesis!
\newcommand{\package}[1]{\textbf{#1}} % package names in bold text
\newcommand{\cmmd}[1]{\textbackslash\texttt{#1}} % command name in tt font 
\newcommand{\href}[1]{#1} % does nothing, but defines the command so the
    % print-optimized version will ignore \href tags (redefined by hyperref pkg).
%\newcommand{\texorpdfstring}[2]{#1} % does nothing, but defines the command
% Anything defined here may be redefined by packages added below...

% This package allows if-then-else control structures.
\usepackage{ifthen}
\newboolean{PrintVersion}
\setboolean{PrintVersion}{false} 
% CHANGE THIS VALUE TO "true" as necessary, to improve printed results for hard copies
% by overriding some options of the hyperref package below.

%\usepackage{nomencl} % For a nomenclature (optional; available from ctan.org)
\usepackage{amsmath,amssymb,amstext} % Lots of math symbols and environments
\usepackage[pdftex]{graphicx} % For including graphics N.B. pdftex graphics driver 

% Hyperlinks make it very easy to navigate an electronic document.
% In addition, this is where you should specify the thesis title
% and author as they appear in the properties of the PDF document.
% Use the "hyperref" package 
% N.B. HYPERREF MUST BE THE LAST PACKAGE LOADED; ADD ADDITIONAL PKGS ABOVE
\usepackage[pdftex,letterpaper=true,pagebackref=false]{hyperref} % with basic options
		% N.B. pagebackref=true provides links back from the References to the body text. This can cause trouble for printing.
\hypersetup{
    plainpages=false,       % needed if Roman numbers in frontpages
    pdfpagelabels=true,     % adds page number as label in Acrobat's page count
    bookmarks=true,         % show bookmarks bar?
    unicode=false,          % non-Latin characters in Acrobat’s bookmarks
    pdftoolbar=true,        % show Acrobat’s toolbar?
    pdfmenubar=true,        % show Acrobat’s menu?
    pdffitwindow=false,     % window fit to page when opened
    pdfstartview={FitH},    % fits the width of the page to the window
    pdftitle={uWaterloo\ LaTeX\ Thesis\ Template},    % title: CHANGE THIS TEXT!
    pdfauthor={Weining Liu},    % author: CHANGE THIS TEXT! and uncomment this line
    pdfsubject={Software Test Selection Frameworkd},  % subject: CHANGE THIS TEXT! and uncomment this line
%    pdfkeywords={keyword1} {key2} {key3}, % list of keywords, and uncomment this line if desired
    pdfnewwindow=true,      % links in new window
    colorlinks=true,        % false: boxed links; true: colored links
    linkcolor=blue,         % color of internal links
    citecolor=green,        % color of links to bibliography
    filecolor=magenta,      % color of file links
    urlcolor=cyan           % color of external links
}
\ifthenelse{\boolean{PrintVersion}}{   % for improved print quality, change some hyperref options
\hypersetup{	% override some previously defined hyperref options
%    colorlinks,%
    citecolor=black,%
    filecolor=black,%
    linkcolor=black,%
    urlcolor=black}
}{} % end of ifthenelse (no else)

% Setting up the page margins...
% uWaterloo thesis requirements specify a minimum of 1 inch (72pt) margin at the
% top, bottom, and outside page edges and a 1.125 in. (81pt) gutter
% margin (on binding side). While this is not an issue for electronic
% viewing, a PDF may be printed, and so we have the same page layout for
% both printed and electronic versions, we leave the gutter margin in.
% Set margins to minimum permitted by uWaterloo thesis regulations:
\setlength{\marginparwidth}{0pt} % width of margin notes
% N.B. If margin notes are used, you must adjust \textwidth, \marginparwidth
% and \marginparsep so that the space left between the margin notes and page
% edge is less than 15 mm (0.6 in.)
\setlength{\marginparsep}{0pt} % width of space between body text and margin notes
\setlength{\evensidemargin}{0.125in} % Adds 1/8 in. to binding side of all 
% even-numbered pages when the "twoside" printing option is selected
\setlength{\oddsidemargin}{0.125in} % Adds 1/8 in. to the left of all pages
% when "oneside" printing is selected, and to the left of all odd-numbered
% pages when "twoside" printing is selected
\setlength{\textwidth}{6.375in} % assuming US letter paper (8.5 in. x 11 in.) and 
% side margins as above
\raggedbottom

% The following statement specifies the amount of space between
% paragraphs. Other reasonable specifications are \bigskipamount and \smallskipamount.
\setlength{\parskip}{\medskipamount}

% The following statement controls the line spacing.  The default
% spacing corresponds to good typographic conventions and only slight
% changes (e.g., perhaps "1.2"), if any, should be made.
\renewcommand{\baselinestretch}{1} % this is the default line space setting

% By default, each chapter will start on a recto (right-hand side)
% page.  We also force each section of the front pages to start on 
% a recto page by inserting \cleardoublepage commands.
% In many cases, this will require that the verso page be
% blank and, while it should be counted, a page number should not be
% printed.  The following statements ensure a page number is not
% printed on an otherwise blank verso page.
\let\origdoublepage\cleardoublepage
\newcommand{\clearemptydoublepage}{%
  \clearpage{\pagestyle{empty}\origdoublepage}}
\let\cleardoublepage\clearemptydoublepage

%======================================================================
%   L O G I C A L    D O C U M E N T -- the content of your thesis
%======================================================================
\begin{document}

% For a large document, it is a good idea to divide your thesis
% into several files, each one containing one chapter.
% To illustrate this idea, the "front pages" (i.e., title page,
% declaration, borrowers' page, abstract, acknowledgements,
% dedication, table of contents, list of tables, list of figures,
% nomenclature) are contained within the file "uw-ethesis-frontpgs.tex" which is
% included into the document by the following statement.
%----------------------------------------------------------------------
% FRONT MATERIAL
%----------------------------------------------------------------------
% T I T L E   P A G E
% -------------------
% Last updated May 24, 2011, by Stephen Carr, IST-Client Services
% The title page is counted as page `i' but we need to suppress the
% page number.  We also don't want any headers or footers.
\pagestyle{empty}
\pagenumbering{roman}

% The contents of the title page are specified in the "titlepage"
% environment.
\begin{titlepage}
        \begin{center}
        \vspace*{1.0cm}

        \Huge
        {\bf University of Waterloo E-Thesis Template for \LaTeX }

        \vspace*{1.0cm}

        \normalsize
        by \\

        \vspace*{1.0cm}

        \Large
        Pat Neugraad \\

        \vspace*{3.0cm}

        \normalsize
        A thesis \\
        presented to the University of Waterloo \\ 
        in fulfillment of the \\
        thesis requirement for the degree of \\
        Master of Science \\
        in \\
        Zoology \\

        \vspace*{2.0cm}

        Waterloo, Ontario, Canada, 2007 \\

        \vspace*{1.0cm}

        \copyright\ Pat Neugraad 2007 \\
        \end{center}
\end{titlepage}

% The rest of the front pages should contain no headers and be numbered using Roman numerals starting with `ii'
\pagestyle{plain}
\setcounter{page}{2}

\cleardoublepage % Ends the current page and causes all figures and tables that have so far appeared in the input to be printed.
% In a two-sided printing style, it also makes the next page a right-hand (odd-numbered) page, producing a blank page if necessary.
 


% D E C L A R A T I O N   P A G E
% -------------------------------
  % The following is the sample Delaration Page as provided by the GSO
  % December 13th, 2006.  It is designed for an electronic thesis.
  \noindent
I hereby declare that I am the sole author of this thesis. This is a true copy of the thesis, including any required final revisions, as accepted by my examiners.

  \bigskip
  
  \noindent
I understand that my thesis may be made electronically available to the public.

\cleardoublepage
%\newpage

% A B S T R A C T
% ---------------

\begin{center}\textbf{Abstract}\end{center}

This is the abstract.

Vulputate minim vel consequat praesent at vel iusto et, ex delenit, esse euismod luptatum augue ut sit et eu vel augue autem feugiat, quis ad dolore. Nulla vel, laoreet lobortis te commodo elit qui aliquam enim ex iriure ea ullamcorper nostrud lorem, lorem laoreet eu ex ut vel in zzril wisi quis. Nisl in autem praesent dignissim, sit vel aliquam at te, vero dolor molestie consequat.

Tation iriure sed wisi feugait odio dolore illum duis in accumsan velit illum consequat consequat ipsum molestie duis duis ut ullamcorper. Duis exerci odio blandit vero dolore eros odio amet et nisl in nostrud consequat iusto eum suscipit autem vero. Iusto dolore exerci, ut erat ex, magna in facilisis duis amet feugait augue accumsan zzril delenit aliquip dignissim at. Nisl molestie nibh, vulputate feugait nibh luptatum ea delenit nostrud dolore minim veniam odio volutpat delenit nulla accumsan eum vero ullamcorper eum. Augue velit veniam, dolor, exerci ea feugiat nulla molestie, veniam nonummy nulla dolore tincidunt, consectetuer dolore nulla ipsum commodo.

At nostrud lorem, lorem laoreet eu ex ut vel in zzril wisi. Suscipit consequat in autem praesent dignissim, sit vel aliquam at te, vero dolor molestie consequat eros tation facilisi diam dolor. Odio luptatum dolor in facilisis et facilisi et adipiscing suscipit eu iusto praesent enim, euismod consectetuer feugait duis. Odio veniam et iriure ad qui nonummy aliquip at qui augue quis vel diam, nulla. Autem exerci tation iusto, hendrerit et, tation esse consequat ut velit te dignissim eu esse eros facilisis lobortis, lobortis hendrerit esse dignissim nisl. Nibh nulla minim vel consequat praesent at vel iusto et, ex delenit, esse euismod luptatum.

Ut eum vero ullamcorper eum ad velit veniam, dolor, exerci ea feugiat nulla molestie, veniam nonummy nulla. Elit tincidunt, consectetuer dolore nulla ipsum commodo, ut, at qui blandit suscipit accumsan feugiat vel praesent. In dolor, ea elit suscipit nisl blandit hendrerit zzril. Sit enim, et dolore blandit illum enim duis feugiat velit consequat iriure sed wisi feugait odio dolore illum duis. Et accumsan velit illum consequat consequat ipsum molestie duis duis ut ullamcorper nulla exerci odio blandit vero dolore eros odio amet et.

In augue quis vel diam, nulla dolore exerci tation iusto, hendrerit et, tation esse consequat ut velit. Duis dignissim eu esse eros facilisis lobortis, lobortis hendrerit esse dignissim nisl illum nulla minim vel consequat praesent at vel iusto et, ex delenit, esse euismod. Nulla augue ut sit et eu vel augue autem feugiat, quis ad dolore te vel, laoreet lobortis te commodo elit qui aliquam enim ex iriure. Ut ullamcorper nostrud lorem, lorem laoreet eu ex ut vel in zzril wisi quis consequat in autem praesent dignissim, sit vel. Dolore at te, vero dolor molestie consequat eros tation facilisi diam. Feugait augue luptatum dolor in facilisis et facilisi et adipiscing suscipit eu iusto praesent enim, euismod consectetuer feugait duis vulputate veniam et.

Ad eros odio amet et nisl in nostrud consequat iusto eum suscipit autem vero enim dolore exerci, ut. Esse ex, magna in facilisis duis amet feugait augue accumsan zzril. Lobortis aliquip dignissim at, in molestie nibh, vulputate feugait nibh luptatum ea delenit nostrud dolore minim veniam odio. Euismod delenit nulla accumsan eum vero ullamcorper eum ad velit veniam. Quis, exerci ea feugiat nulla molestie, veniam nonummy nulla. Elit tincidunt, consectetuer dolore nulla ipsum commodo, ut, at qui blandit suscipit accumsan feugiat vel praesent.

Dolor zzril wisi quis consequat in autem praesent dignissim, sit vel aliquam at te, vero. Duis molestie consequat eros tation facilisi diam dolor augue. Dolore dolor in facilisis et facilisi et adipiscing suscipit eu iusto praesent enim, euismod consectetuer feugait duis vulputate.

\cleardoublepage
%\newpage

% A C K N O W L E D G E M E N T S
% -------------------------------

\begin{center}\textbf{Acknowledgements}\end{center}

I would like to thank all the little people who made this possible.
\cleardoublepage
%\newpage

% D E D I C A T I O N
% -------------------

\begin{center}\textbf{Dedication}\end{center}

This is dedicated to the one I love.
\cleardoublepage
%\newpage

% T A B L E   O F   C O N T E N T S
% ---------------------------------
\renewcommand\contentsname{Table of Contents}
\tableofcontents
\cleardoublepage
\phantomsection
%\newpage

% L I S T   O F   T A B L E S
% ---------------------------
\addcontentsline{toc}{chapter}{List of Tables}
\listoftables
\cleardoublepage
\phantomsection		% allows hyperref to link to the correct page
%\newpage

% L I S T   O F   F I G U R E S
% -----------------------------
\addcontentsline{toc}{chapter}{List of Figures}
\listoffigures
\cleardoublepage
\phantomsection		% allows hyperref to link to the correct page
%\newpage

% L I S T   O F   S Y M B O L S
% -----------------------------
% To include a Nomenclature section
% \addcontentsline{toc}{chapter}{\textbf{Nomenclature}}
% \renewcommand{\nomname}{Nomenclature}
% \printglossary
% \cleardoublepage
% \phantomsection % allows hyperref to link to the correct page
% \newpage

% Change page numbering back to Arabic numerals
\pagenumbering{arabic}

 

%----------------------------------------------------------------------
% MAIN BODY
%----------------------------------------------------------------------
% Because this is a short document, and to reduce the number of files
% needed for this template, the chapters are not separate
% documents as suggested above, but you get the idea. If they were
% separate documents, they would each start with the \chapter command, i.e, 
% do not contain \documentclass or \begin{document} and \end{document} commands.
%======================================================================
%\include{chapters/chapter1/chapter1}
\chapter{introduction}\label{chap:intro}


Regression Test Selection(RTS) is a research area aimed at selecting a subset
of test cases from exsiting test suite for regression testing. Many RTS
techniques have been proposed and evaluated , %1)test selection is effective
empirical evaluation of these technqiues indicats that algorithms can be very
effective in reducing the size of the test suite while still maintain
safty\cite{}. One of the challenges for test organizations adopt RTS techniques
is to determine which specific RTS technique to use. Our proposed STS framework helps test engineers explictly express their test selection goals, systematically analyze software artifacts available in the development process, and select a STS technique provides the most benifits with lowest cost. This framework also helps the research community access to published STS techniques and reproduce the experiments or
conduct new experiments.




2)effectiveness of each technqiues varies
 these studies also show that the facors such as the structure of the
 code,location of the changes, granularity of hte test suite, and frequency of
 the testing affect the reduciton in test suite size that can be achieved by the
 technqiues.

3)how engineers select which techqniues to use? 
Testing is a costly process that can yield over half of total
devopment cost. Regardless the investment on testing, exhaustive testing is not
practical in the industry. In most industry software projects, some forms of
test selection do exist. However, the  Traditionally, decision on which test
selection technqiues to use is made mostly ad hoc. Test engineers may choose a
particular technique based on previous empirical evaluation results of the
technique. However, as empirical evaluation is high dependent on the test
subjects. \cite{DBLP:conf/esem/KasurinenTS10}


On the other hand, it does not make economical sense if the cost of choosing a
good test selection technqiue is higher than the cost of test selection itself.
Or more precisely, the benefit received from test selection should higher
than the cost of the test selection , including cost of choosing a
test selection techqniue and applying the tecnique.
\underline{need to find out an abbreviation of software test selection
technqiues - STS techniques?}


a safe regression test selection technique is one that , under certain
assumptions, selects every test case from the original test suite that can
expose fault in the modified program.\cite{DBLP:journals/tse/RothermelH96}

To our knowledge, this paper is the first to propose a framework to select STS
technqiues. Most of the STS researches to date are focusing on
finding an effective STS technqiue and performing post morterm empirical
evaluation of techqniues. These focus are not surprising from academia's point of
view as the goal of the research is to find an effective techqnique.
However,from engineering point of view, the first challegne is to determine
which STS technqiue to use. 

\chapter{Releated Work}\label{ch:related-work}
%======================================================================
In this chapter, we survey research results related to Test Case Selection
problem. Various approaches have been proposed using different
techniques. Similar to \cite{Yoo:2010}, we surveryed literature has been
published in major software engineering domain, especially in software testing and softwar
maintenance literature. Different from \cite{Yoo:2010}, our focus is on software
test selection. Though some minimization and prioritization literatures also share
similar goals and technqiues as STS. Another difference from \cite{Yoo:2010}is
that we first group the literatures by their goals and required data, then group by
technqiues. This way is more aligned with the determination process in our
framework. The following subsections describe these literatures in details and
highlighting their strengths and weaknesses.

Arafeen and Do\cite{DBLP:conf/issre/ArafeenD11} applied Analytic Hierarchy
Process(AHP) method for selecting test prioritization
techniques.\underline{talk more about how it's done, what is AHP etc}. It's
closely related to our research as it's also trying to address the problem of selecting a technique for test practitioners.The empirical study results show AHP is effective for
selecting appropriate test case prioritization techniques across
system lifetime. AHP outperformed 15 out of 20 cases when the total cost-benefit
values are considered, AHP outperformed 16 out of 20 cases when total number of version that performed best is considered.
However, the pairwise comparison values in AHP is subject to human judgement
and thus the results can be biased by personnel's knowledge and experience.
In the empirical study of \cite{DBLP:conf/issre/ArafeenD11}, values in AHP are
assigned based on history data regarding the performance of test case prioritization techniques observed
from previous empirical studies. However, empirical evidence for differences
between techniques is not very strong, and sometimes contradictory\cite{}. Also
practicality point of view,  assigning values in AHP could be a daunting task
for a test practitioner who do not have to the access or the knowledge of
previous empirical studies in academia. 

%----------------------------------------------------------------------

\section{backgroud}

\section{Code modification based techniques}
%----------------------------------------------------------------------

Many researchers proposed techniques focusing on selecting all nonobsolete tests
in $T$ that execute changed code from $P$ to $P^\prime$. These techniques are
called safe techniques as they select all modification-traversing test cases in
$T$. As $T_{fr} = T_{mr} \le T_{mt}$, these techniques most likely are not
precise, i.e. not all selected modification-traversing test cases will reveal
fault. These techniques typically require knowledge of code coverage information
of each test case $t$ in $T$ and source code of $P$ and $P^\prime$.




%See equation \ref{eqn_pi} on page \pageref{eqn_pi}.\footnote{A famous
%equation.}


\newpage
\section{Cost Models}
Do et al. has proposed and applied comprehensive economic model for
regression testing process
~\cite{DBLP:conf/issta/DoR08,DBLP:conf/sigsoft/DoMTR08,DBLP:conf/sigsoft/DoR06,5482587}.
This cost model captures both cost and benefit factors relavant to regression testing process. The cost model is used to support evaluation of various
regression test selection techniques.
\\The major difference between our cost
model and cost model presented in \cite{DBLP:conf/issta/DoR08} is that their
cost model is post mortem, i.e the cost model is to evaluate the effect of
regression testing techniques after the regression testing phase completed.
Our cost model is to provide cost estimation prior to apply any test selection technique. As suggested in \cite{DBLP:conf/sigsoft/DoR06}, test selection technique might vary with test suite type. Providing cost estimation to practioners up front will help them to make consicous decision on which test selection technique to choose for a specific release or test suite.
Our cost model is based on \cite{DBLP:conf/issta/DoR08} and has removed factors
can not be abtained or resonablly estimated before regression testing
start.Specificially, we made following changes to Do's cost
model.\begin{itemize}\item Remove Delayed fault detection feedback(CD).CD
captures the cost of delaying product release due to late discovered faults. CD directly concerns with order of the test cases.Cost of delayed fault feedback would be low if test cases are
executed in a priority order such that faults are found early in the regression
testing cycle and sufficent time is given to fix the faults prior to
release date. However, our cost model is to provide cost estimation on
various test selection techniques. We do not concerns with the order of test
cases therefore we removed this cost factor from our model.\item Do's cost
model accounts for lifetime factors by tracking costs and benefits across entire
sequences of system releases. However, it's very difficult to estimate the
number of release $n$ prior to regression testing phase. There are research in
predicting total number of faults in a
release\cite{DBLP:journals/smr/RajaHH09}. However, the number of releases
built in regression testing phase may not directly related to total number of faults corrected. Some organizations create new bulid on a time boxed schedule, e.g one build per week; some organizations create new build on demand, e.g. only when there are significant faults are corrected a new build will
be created. In our experiement, we assume a fixed daily build. \item Consider
the cost of Known Fault.\underline{Explain different test execution status
passed,failed, blocked}. When there are difference resources to fix faults and
\end{itemize}

\section {A surver of software test optimization goals}
\underline{a comprehensive list of all papers and what are the goals,organize
based on goals and subgoals,incl. selelction,prioritzation and reduction}. 
Many software test optimization techniques including test selection,
prioritization and reductin technqiues have been researched over the
years.Researchers normally have one or many goals implcitly or explictly in mind
when researching techqniuqes. In this section, we survery the goals behind
current software test optimizatioin researchs to date and point out how these
goals impact our framework design.


\section {Goal Modelling}
 %3 ways to model goals- AND-OR graph,SIG, Lattice , what is goal modelling in
 % general. what is SIG, what others use and why we don't use it.
Goal Modelling is an important element in requirments engineering to visulize
functional or non-funcitonal requirements. It's also used widely in Process Analysis, Business Analysis and other engineering domains. A goal in Requirement Engineering is a stakeholder's objective for the system. A goal relates to one or many low-level requirements. A goal
model is a set of nodes representing hierarchically decomposed goals plus a
set of edges representing both decomposition and contribution relationships
among goals\cite{Ernst:2006:VNR:1173697.1174072}.Goal modeling helps engineers systematically formalize stakeholder goals and visulaze hireacy and relationship of goals.

Goal models can also be used to capture actors and dependencies
\cite{Ernst:2006:VNR:1173697.1174072,DBLP:conf/icse/YuM94} An Actor Dependecny
model consists of a set of nodes to represent actors , and each link between two
actors indicates that one actor depends on the other in order that
the former may attain some goals.
 
Salehie et al. use GQM approach to identify goals and  metrics for test case
priroitziation. GQM does not capature interdependcey between goals. 
 
Soft goals are goals can be partially achieved.


%May 30,2011
Chung et al. proposed a Softgoal Interdependecy Graphs in \cite{NFR} as part of
the non-functional requirement framework. The major concepts of the framework is
a graph representation of softgoals and interdependices among softgoals.
In SIG, Softgoals, which are ``soft'' in nature, are shown as clouds. Softgoals have associated labels (values representing the degree to
which a softgoal is achieved) which are used to support the reasoning process
during design. Softgoals can also have an associated prioirty
(criticality or dominace). Softgoals can also have other attributes, such as
author, and time of creation of the softgoal.
Softgoals are connected by interdependecy links, which are shown as lines, often
with arrowheads. Interdependencies show refinements of ``parent'' softgoals downwards into other, more specific, ``offspring'' softgoals. They also show the contribution (impact) of offspring upwards upon the meeting of other (parent) softgoals.

In the following, we provide more details softgoal interdependancies about SIG
based on \cite{NFR}.



In the context of this paper, we use goal models to capture
stakeholders motiviations and intentions to perform test selection. A goal is a
stakeholer's objective for performing test case selection. Stakeholder typically are test engineers or test managers who wish to perform test selection with some objectives in mind. A goal relates to one or
many test selection techniques. 
Engineers select test cases with some test selection technqiues need to consider
the needs and wants from other stakeholders in the organization. Goal modeling
can help engineers to have deeper understanding of various goals from other
software engineers, managers and customers in order to choose the best test
selection techniques and ultimately deliver the best product balance all
stakeholders needs.











\chapter{Introduction}
We expect this framework to help test engineers make better choice of STS
techniques that are most apporiate for their organization and processes.
Challeges- empirical results various for different program.

% ======================================================================
\chapter{3}
% ======================================================================


\section{ Constraint satisfaction problem}
%\href{http://en.wikipedia.org/wiki/Constraint_satisfaction_problem}{Constraint
%satisfaction problem}

\section{Design of Experiments}
In the design of experiments, an idependent variable's values are controlled or
selected by the experimenter to determine its relationship to an observed
phenomenon (i.e. the dependent variable). 

In summary:
Independent variables answer the question "What do I change?"
Dependent variables answer the question "What do I observe?"
Controlled variables answer the question "What do I keep the same?"
Extraneous variables answer the question "What uninteresting variables might mediate the effect of the IV on the DV?"
This would be a good place for some figures and tables.

% \include{chapters/chapter3/chapter3}
% UW Thesis Template for LaTeX 
% Last Updated July 8, 2009 by Stephen Carr, IST Client Services
% FOR ASSISTANCE, please send mail to rt-IST-CSmathsci@ist.uwaterloo.ca

% Effective October 2006, the University of Waterloo 
% requires electronic thesis submission. See the UW thesis regulations at
% http://www.grad.uwaterloo.ca/Thesis_Regs/thesistofc.asp.
% However, many faculties/departments also require one or more printed
% copies. This template attempts to satisfy requirements for both types of output. 
% It is based on the standard "report" document class. The "book" document 
% class can be substituted if you have a very large, multi-part thesis.

% DISCLAIMER
% To the best of our knowledge, this template satisfies the current UW requirements.
% However, it is your responsibility to assure that you have met all 
% requirements of the University and your particular department.
% Many thanks to Colin Alie for assistance in preparing this updated template.

% -----------------------------------------------------------------------

% By default, output is produced that is geared toward generating a PDF 
% version optimized for viewing on an electronic display, including 
% hyperlinks within the PDF.
 
% E.g. to process a thesis called "mythesis.tex" based on this template, run:

% pdflatex mythesis	-- first pass of the pdflatex processor
% bibtex mythesis	-- generates bibliography from .bib data file(s) 
% pdflatex mythesis	-- fixes cross-references, bibliographic references, etc
% pdflatex mythesis	-- fixes cross-references, bibliographic references, etc

% If you use the recommended LaTeX editor, Texmaker, you would open the mythesis.tex
% file, then click the pdflatex button. Then run BibTeX (under the Tools menu).
% Then click the pdflatex button two more times. If you have an index as well,
% you'll need to run MakeIndex from the Tools menu as well, before running pdflatex
% the last two times.

% N.B. The "pdftex" program allows graphics in the following formats to be
% included with the "\includegraphics" command: PNG, PDF, JPEG, TIFF
% Tip 1: Generate your figures and photos in the size you want them to appear
% in your thesis, rather than scaling them with \includegraphics options.
% Tip 2: Any drawings you do should be in scalable vector graphic formats:
% SVG, PNG, WMF, EPS and then converted to PNG or PDF, so they are scalable in
% the final PDF as well.
% Tip 3: Photographs should be cropped and compressed so as not to be too large.

% To create a PDF output that is optimized for double-sided printing: 
%
% 1) comment-out the \documentclass statement in the preamble below, and
% un-comment the second \documentclass line.
%
% 2) change the value assigned below to the boolean variable
% "ElectronicVersion" from "true" to "false".

% --------------------- Start of Document Preamble -----------------------

% Specify the document class, default style attributes, and page dimensions
% For hyperlinked PDF, suitable for viewing on a computer, use this:
%----------------------------------------------------------------------
% MAIN BODY
%----------------------------------------------------------------------
% Because this is a short document, and to reduce the number of files
% needed for this template, the chapters are not separate
% documents as suggested above, but you get the idea. If they were
% separate documents, they would each start with the \chapter command, i.e, 
% do not contain \documentclass or \begin{document} and \end{document} commands.
%======================================================================
\reversemarginpar
\setlength{\marginparwidth}{2cm}
\chapter{Modelling Goals}
%======================================================================
We use exiting goal modelling research to help us identify most apporapriate
metrics for selecting test cases. Though Software Test Selection 
\newpage
%----------------------------------------------------------------------
\section{Software Test Selection Goals}
%----------------------------------------------------------------------

\underline{why we talk about goals, different goals lead to differnt
techniques}Software Test Selection may be
performed for vaiours goals.Differences in underlying goals lead regresstion test selection
techniques to distinctly different results in test selection\cite{DBLP:journals/tse/RothermelH96}. 

Consider a conintuous integration development environment in which new or
changed code are integrated into existing code frequenly and an automated build
task is schedule nightly. In this case, test selection may be used to select a
small subset of existing test suite to test newly introduced or change code
only. Ideally , this selected subset is much smaller than the complete test
suite. If the new or changed code is to implmenent a new requirement or to
implment change to existing requirement, test selection may be used to select a
subset to test new or changed requirements.If the new or changed code is to
implmenent a fix of a defect, test selection may be used to select a subset to
test the fix. 

combination of multiple goals. conflict goals.




\cite{non-functional requirements in software engineering}
softgoal interdependency graphs

unlike goals in traditional problem-solving and planning frameworks,
non-functional requirmenents can rarely be said to be ``accomplished'' or
``satisfied'' in a clear-cut sense. Instead, differnet design decisions
contribute positively or negatively, and with variying degreees, towards
attaining a particular non-functional requirement. Accordingly, we speak of the
satisficing of a softgoals to suggest that generated software is expected to
satisfy NFRs within acceptable limits, rather than absolutely.


offspring can contribute fully or partially , and positiviely or negatively,
towards the satisficing of the parent.





\newpage
A very important feature of our framework are the catalogues of test
selection goals and test selection techniques based on research literautres to date. We
created two kinds of catalogues in our test selection framework. First, we
surveryed existing software test selection,prioritization and minimization papers and created a catalog of software test selection goals. Then, we
systematically organie test selection techniques discuss in the literature.
These techniques are intend to help meet selection goals. With these two
catalogues, we created a SIG to capture the interdependecies
between software test selection goals and relationships between selection goals
and selection techniques. These two catalogues not only the first step for test engineers to pick the right selection techniques,
but also are the fundmental knowledge base for test engineers to understand what
other selection options are available out there. In this chapter, we will discuss in details how
this SIG is created.

%document how I come up with the graph. how the survey was done ans what
% information is collection and compared sysmetically
\subsection{Acquiring and cataloguing Software Test Selection goals}
We followed the decomposing process in \cite{NFR} to decompose our top level
goal. The initial top level softgoal is pretty broad,it needs to be broken down
into smaller subgoals, until a metrics and a selection technqiue can be
found.Together, subgoals ``satisfice'' (should meet) the higher level softgoals.

The catalogues of test selection goals and selection technqiues are composed
from many sources. Once source is  that we have survey test selection papers
published in the major software engineering venue since year 1995, especially in
software testing and softwar maintenance literature. Since there are some
overlaps between test selection and prioritization,minimization, we also extend our survery to prioritization and minimization in some cases. A
complete list of paper survery to build the catagloue is in appendix A.
In surveying test selection releated literatures, we extract information deplict
in figure xx for each paper. Some of the information are not explictly mentioned
in the paper. For example,  most of the tesst optimization research were
performed with one or several goals in mind, but not all goals are specified
explictily. In that case, we have documented the goals for the paper  based on
our understanding of the research. 

\underline{add diagram here goal-data-metrics-evaluation metrics etc. on
whiteboard}
\subsection{Identified test selection goals}


\subsection{summary}
Some selection goals and techniques are also come from the knowledge in the
industry. As this framework is used in practice, test engineers can add
additional goals and techniques to the catalogues in the framework. Knowledge
and experience from wide range of industry settings can be recorded in the
framework and made available for sharing. Test engineers and researcher would
have access to selection goals and techniques beyond their immediate domain ,
and can adapt them to meet the needs for particular domain or project.




In addition, some of the goals are ambigous or not granaluar enough. 


%----------------------------------------------------------------------
% END MATERIAL
%----------------------------------------------------------------------
% B I B L I O G R A P H Y
% -----------------------

% The following statement selects the style to use for references.  It controls the sort order of the entries in the bibliography and also the formatting for the in-text labels.
%\bibliographystyle{plain}
% This specifies the location of the file containing the bibliographic information.  
% It assumes you're using BibTeX (if not, why not?).
%\bibliography{../../bib/thesis}
% The following statement causes the title "References" to be used for the biliography section:
%\renewcommand{\bibname}{References}
%\addcontentsline{toc}{chapter}{\textbf{References}}
% Tip 5: You can create multiple .bib files to organize your references. 
% Just list them all in the \bibliogaphy command, separated by commas (no spaces).

% The following statement causes the specified references to be added to the bibliography% even if they were not 
% cited in the text. The asterisk is a wildcard that causes all entries in the bibliographic database to be included (optional).
%\nocite{*}

% \include{chapters/chapter5/chapter5}
% UW Thesis Template for LaTeX 
% Last Updated July 8, 2009 by Stephen Carr, IST Client Services
% FOR ASSISTANCE, please send mail to rt-IST-CSmathsci@ist.uwaterloo.ca

% Effective October 2006, the University of Waterloo 
% requires electronic thesis submission. See the UW thesis regulations at
% http://www.grad.uwaterloo.ca/Thesis_Regs/thesistofc.asp.
% However, many faculties/departments also require one or more printed
% copies. This template attempts to satisfy requirements for both types of output. 
% It is based on the standard "report" document class. The "book" document 
% class can be substituted if you have a very large, multi-part thesis.

% DISCLAIMER
% To the best of our knowledge, this template satisfies the current UW requirements.
% However, it is your responsibility to assure that you have met all 
% requirements of the University and your particular department.
% Many thanks to Colin Alie for assistance in preparing this updated template.

% -----------------------------------------------------------------------

% By default, output is produced that is geared toward generating a PDF 
% version optimized for viewing on an electronic display, including 
% hyperlinks within the PDF.
 
% E.g. to process a thesis called "mythesis.tex" based on this template, run:

% pdflatex mythesis	-- first pass of the pdflatex processor
% bibtex mythesis	-- generates bibliography from .bib data file(s) 
% pdflatex mythesis	-- fixes cross-references, bibliographic references, etc
% pdflatex mythesis	-- fixes cross-references, bibliographic references, etc

% If you use the recommended LaTeX editor, Texmaker, you would open the mythesis.tex
% file, then click the pdflatex button. Then run BibTeX (under the Tools menu).
% Then click the pdflatex button two more times. If you have an index as well,
% you'll need to run MakeIndex from the Tools menu as well, before running pdflatex
% the last two times.

% N.B. The "pdftex" program allows graphics in the following formats to be
% included with the "\includegraphics" command: PNG, PDF, JPEG, TIFF
% Tip 1: Generate your figures and photos in the size you want them to appear
% in your thesis, rather than scaling them with \includegraphics options.
% Tip 2: Any drawings you do should be in scalable vector graphic formats:
% SVG, PNG, WMF, EPS and then converted to PNG or PDF, so they are scalable in
% the final PDF as well.
% Tip 3: Photographs should be cropped and compressed so as not to be too large.

% To create a PDF output that is optimized for double-sided printing: 
%
% 1) comment-out the \documentclass statement in the preamble below, and
% un-comment the second \documentclass line.
%
% 2) change the value assigned below to the boolean variable
% "ElectronicVersion" from "true" to "false".

% --------------------- Start of Document Preamble -----------------------

% Specify the document class, default style attributes, and page dimensions
% For hyperlinked PDF, suitable for viewing on a computer, use this:
%----------------------------------------------------------------------
% MAIN BODY
%----------------------------------------------------------------------
% Because this is a short document, and to reduce the number of files
% needed for this template, the chapters are not separate
% documents as suggested above, but you get the idea. If they were
% separate documents, they would each start with the \chapter command, i.e, 
% do not contain \documentclass or \begin{document} and \end{document} commands.
%======================================================================
\reversemarginpar
\setlength{\marginparwidth}{2cm}
\chapter{Proposing Selection Techniques}
\label{chap:Proposing Selection Techniques}
%======================================================================
\section{Precision Predictor}\label{sec:Precision Predictor}
\underline{talk about why do we need precision predictor, rw, and others}

Precision] measures the extent to which a technique select
l  test cases that are non-modification-revealing.

\section{Cost Predictor}
\label{sec:Cost Predictor}
\underline{focus on how to predict analysis cost, rather than why need to make
the prediction}

\underline{an updated technique modelling diagram showing implementation has
cost and precision predictor}
%----------------------------------------------------------------------
\section{Techniques Determination}
\label{sec:Techniques Determination}
In section~\ref{sec:Precision Predictor}, we discussed Precision
Predictor component in our Framework. The Precision Predictor component provides
valuable precision prediction for researchers or test engineers to compare one
technique over another.  However, it's likely that precision is not the only metric to
determine a test selection technique as technique with lower precision value
(ie. less percentage of test cases selected from original test suite) is not
always a better technique. To identify what is a better technique, initially
researchers have proposed analytical approaches to access several attributes
of techniques and compare techniques\cite{DBLP:journals/tse/RothermelH96}.
\begin{description}
\item[Precision] measures the extent to which a technique select
test cases that are non-modification-revealing.
\item[Inclusiveness] measures the extent to which a technique
omits tests in $T$ that reveal faults in a modified program. a 100 percent
inclusive technique is safe.
\item[Efficiency] measures the space and time requirements of a
technique.
\item[Generality] measures the ability of a technique to function in a practical
and sufficiently wide range of situations.
\end{description} 

\underline{Q:These factors are good to consider when evaluate STS techniques
post-mortem. Can we actually use them prior to regression test? Can we use
evaluation results from previous experiments/history values?}%
These attributes help to compare the strength and weakness of techniques in
general, however it's a qualitative analysis rather than quantitative. Our work
goes beyond goal-based STS techniques by allowing researchers and test
engineers to use a utility function to determine which technique to use for the
application under test. Our utility function incorporate the attributes
mentioned about and produce one single utility value.

Utility is a value that represents the desirability of a particular state or
outcome.\underline{need a better reference to utility here}
%\href{http://www.pdl.cmu.edu/PDL-FTP/Storage/CMU-PDL-08-102.pdf}{utility function section 1.1.3}
The utility function provides the ability to collapse
multiple objectives into a a single axis that can be used to compare with all
candidate STS techniques.Using utility, both the cost and benefits of each STS
technique can be examined in a single framework. Using utility function also
allows an automated optimization system to properly balance cost and benefits of each STS
techniques.

Ultimate goal of any selection is to find a technique gives the highest
benefits with lowest cost. Our utility function predicts cost-benefit on each candidate
techniques based on the output from Precision Predictor and Cost Predictor. Our
utility function is based on cost model presented in 
\cite{DBLP:conf/sigsoft/DoR06} but it's different from it in  
\begin{itemize}
  \item Our Utility Function is applied prior to test selection and test
  execution phase, while D-R cost model is applied after test
  execution phase for post-mortem analysis
  \item Utility value calculated from Utility Function does not represent cost
  in dollar amount. The utility value does not have a measurement unit. It's
  just a single value to compare one technique to another. D-R cost model does
  represent cost in dollars, therefore it requires additional parameters like
  organization's revenue in dollars per time unit, average hourly salary
  associated with employing a programmer per unit of time. Additional cost
  parameters in D-R cost model requires estimation inputs from
  each organization/teams and makes it more complex then our Utility Function.
  \item Cost components constant to any STS techniques are not included in our
  Utility Function as they do not contribute any difference in comparing STS
  techniques. This further simplifies our Utility Function. 
\end{itemize} 

\underline{describe cost model}
\subsection{Cost Components}\label{sec:cost_components}
In the following section, we describe cost components proposed in
D-R cost model\cite{DBLP:conf/sigsoft/DoR06} with emphasis on
which components are included in the Utility Function and how these components
are included. A summary of all cost components in relation to our Utility
Function is listed in table \ref{}. Methods for measuring or estimating these
components in our framework implementation are described in more details in
section~ref{chap:Empirical Evaluations}.
\begin{description}
\item[Test Setup]($CS$). $CS$ includes cost of activities
required to prepare to run tests.This includes costs of setting up test environments,
configuring test drivers and test simulators. This cost is necessary to execute
the test suite regardless if entire test suite is selected or only one test
case is selected. $CS$ is not dependent on selection technique,
therefore it's not included in the Utility Function.
\item[Identifying obsolete test case]($CO_i$). $CO_i$ represents
cost to determine which test cases are still applicable to a new version of the
program. Similar to $CS$, this cost is necessary to execute
the test suite regardless if entire test suite is selected or only one test
case is selected. $CO_i$ is not dependent on selection technique,
therefore it's not included in the Utility Function.
\item[Repairing obsolete test cases]($CO_r$). $CO_r$ represents
cost to repair obsolete test cases for reuse.$CO_r$ is necessary to
execute the test suite regardless if entire test suite is selected or only one
test case is selected. $CO_r$ is not dependent on selection technique,
therefore it's not included in the Utility Function.
\item[Supporting analysis]($CA$). $CA$ represents the cost of
analysis needed to support a STS technique. $CA$ includes cost of
instrumenting code, analyzing changes and collecting test execution traces. In
our Utility Function, this is the predicated analysis cost. As described in
section ref{sec:Cost Predictor}, each unique STS technique in the framework
implements a Cost Predictor to predicate $CA$ value.
\item[Regression testing technique execution]($CR$). $CR$ represents the cost of
applying a STS technique after analysis has been completed. Each unique STS technique in the framework implements a Cost
Predictor to predicate $CR$ value.
\item[Test execution]($CE$). $CE$ represents the cost of executing
all selected test cases. this includes human and machine costs for all manual,
automatic , or semi-automatic test cases. $CE$ is dependent on the number
of test cases selected for regression. If the run time of each test case is
uniformed in a test suite, $CE$ cost is less when less number of
test cases is selected. In our Utility Function, we make simplifying assumption
that each test in the test suite has equal run time. We use $CE_avg$
denotes average execution cost of each test case. 
\item[Test result validation]($CV_d and CV_i$). $CV_d and CV_i$ represent the
cost of checking test results to determine whether or not test case reveal failures. $CV_d$ is the cost of comparing test results
automatically, $CV_i$ is the cost of comparing test results
manually. $CV_d and CV_i$ are dependent on the number of test cases
selected. In our Utility Function, we make simplifying assumption that each test
in the test suite has equal test result validation time. We use $CV_avg$
denotes average test result validation time for each test. $CV_avg$
includes both manual and automated result validation time.
\item[Missing faults]($CF$). $CF$ represents the cost of missing
faults. This includes the costs of missing fault that the test suite could have
detected if executed in full, but missed due to omission of some test cases. If
the STS technique is a safe technique, we do not count $CF$ in the
Utility Function. \underline{really?what is a safe technique}
\item[Delayed fault detection feedback]($CD$). $CD$ is the cost
of delaying product release due to late discovered faults. As a simple example,
if bug fixing cycle is two days but a fault is found one day before scheduled
release time, the release has to be delayed for at least one day in order to
take a new build with the fix of that late discovered fault. This cost could occur
when fault revealing test cases were executed too late in the testing cycle and
high $CD$ cost indicates a test case prioritization issue. Clearly,
$CD$ is dependent on execution orders of the test cases but not dependent
on STS techniques. Test cases selected by the same STS technique could be
executed in an optimized order to incur zero or low $CD$ cost or in an
un-optimized order to incur high $CD$ cost. Therefore, the Utility
Function in our framework do not consider $CD$ cost component.
\end{description}
\underline{discuss which cost in which phase} Recall in section~ref{sec:???}, we
discussed testing phases. 

%% Table generated by Excel2LaTeX from sheet 'TABLE FOR LATEX'
\begin{table}[htbp]
\begin{small}
  \centering
  \caption{Summary of Cost Components in D-R Cost Model and Utility
 Function}
    \begin{tabular}{rrrrrrr}
   % \toprule
    \textbf{Cost} &    & \textbf{Included}
    & \textbf{Included} & \textbf{Technique} &
     & \textbf{Human} \\
    \textbf{Component} & \textbf{Description} & \textbf{in D-R}
    & \textbf{in Utility} & \textbf{dependent} &
    \textbf{Phase} & \textbf{/machine cost} \\
    
    %\midrule
    $CS$ & Test setup & Y & N & N &Pre-regression & Human \\
    $CO_i$ & Identifying obsolete test cases & Y     & N     & N     &
    Pre-regression & Human \\
    $CO_r$ & Repairing obsolete test cases  - & Y     & N     & N     &
    Pre-regression & Human \\
    $CA_{in}$ & Instrumenting all units & Y     & Y     & Y     & Pre-regression
    & Machine \\
    $CA_{tr}$ & Collecting traces for test cases  & Y     & Y   & Y     &
    Pre-regression& Machine \\
    $CR$ & Regression testing technique execution & Y     & Y     & Y    &  Pre-regression & Machine \\
    $CE$ & Test execution & Y     & N     & Y     &   Regression & Human\&Machine \\
    $CE_{avg}$& Test execution per test case & N     & Y     & N   & Regression & Human\&Machine \\
    $CV_d$ and $CV_i$ & Test result validation& Y     & N     & Y     & Regression & Human\&Machine \\
    $CV_{avg}$ & Test result validation per test case & N     & Y     & N     & Regression& Human\&Machine \\
    $CF$ & Missing faults & Y     & Y     & Y     & Post - release  &    Human\&Machine \\
    $CD$ & Delayed fault detection feedback& Y     & N     & N     & Regression & Human\&Machine \\
    %\bottomrule
    \end{tabular}%

\end{small}
\label{tab:Summary_of_Cost_Components} 
\end{table}%


Table~\ref{tab:Summary_of_Cost_Components} summarizes cost components in D-R
cost model and cost components considered in our Utility Function.
\underline{describe more about the table for each column and abbreviation if
not clear}


\pagebreak

\subsection{Utility Function}\label{Utility_function}
The utility function is ultimately a simplified cost model to perform cost
benefit analysis prior to test selection and after test selection. Before we
discuss Utility Function, we define several parameters and coefficients that are
used in the Utility Function, most of which instantiated cost components outlined in section~\ref{sec:cost_components}. Assume we have a candidate list of STS techniques $M_1$,$M_2$,\ldots,$M_k$, $n$ versions of
programs $P$ and $n$ versions of test suite $T$ (one per version of program P).
Set of test cases selected by technique $M$ denoted $T_m$.
\begin{itemize}
  \item $i$ is an index denoting a particular version $P_i$ of $P$.
  \item $k$ is an index denoting a particular technique $M_k$ of $M$.
  \item $CA_{in}(i)$ is the cost of instrument all units in $P_i$.
  \item $CA_{tr}(i)$ is the cost of collecting traces for test cases in
  $T_{i-1}$ over $P_{i-1}$ 
  \item $CR(i)$ is the cost of applying test selection technique $M_i$
  \item $CE_{avg}(i)$ is the average execution cost of one test case in $T_i$
  \item $CV_{avg}(i)$ is the average test result validation cost of one test
  case, including automated and manual tests
  \item $CF$ is the cost of missing faults that the regression test suite
  could have detected if executed in full.
  \item $\theta_c$ ratio of cost in regression phase(critical phase) over
  pre-regression phase (preliminary phase)
  \item $\theta_p$ ratio of cost in post-release phase over pre-regression phase
  (preliminary phase)
   
\end{itemize}

The two equations that comprise our Utility Function are as follows: 
$$ Cost = \sum_{i=2}^{n}(CA_{in}(i)+CA_{tr}(i)+CR(i)+CF(i)*\theta_p)$$ 
$$ Benefit = \sum_{i=2}^n((|T(i)|-|Tm(i)|)*(CE_{avg}(i)+CV_{avg}(i))*\theta_c)$$

Utility Function $$ \digamma = Benefit - Cost $$

The Utility Function is applied  twice in the framework,once in Proposing Selection Technique phase, once in EValation
phase. In the Proposing Selection Technique phase, the Utility Function is
applied to all candidate STS techniques and utility values are calculated.
Technique with the highest utility value is proposed. In Evaluation phase,  the
Utility Function is once again applied to all candidate STS techniques and
utility values are re-calculated. The proposed technique is evaluated against
technique with the highest actual utility value.

One of the assumptions we made in the utility function above is that
every test case selected will always be executed. This may not be true in some
situations especially in organizations with separate test teams and development
teams.A new version of the program could be kicked off by development team while test team has not yet finished execution of the
selected test set $T_m$ on current version. In this case, test team has
several choices based on organization's processes and the utility function can
be applied with some modifications discussed below.
\begin{itemize}
  \item continue to execute the remaining test cases on the current
  version and ignore the new version until all tests are completed. In
  this case, the utility function can be used without any modification.
  \item stop test execution on current version. Reselect test cases based on the
  new version and execute the newly selected test cases on new version. In this
  case,test team essentially performed a unsafe test selection on current
  version. All test cases executed so far against the current version
  are selected;all remaining test cases in the test suite are considered
  discarded. We can adjust the cost component in the utility function so that
  $T_m$ is the set of test cases executed on current version;$CF$ is the
  cost of missing faults due to omitting remaining test cases.
  \item execute the remaining test cases on the new version without reapplying
  test selection technique. In this case, $n$ no longer represents number of
  program versions as test selection is not perform on each every program
  version. Instead, $n$ represents number of times test selection was performed
  in a release. $i$ becomes an index denoting a particular test selection performed over a program version. 
\end{itemize}


\subsection{estimation of utility parameters}
In this section, we discuss in more details how to estimate/measure each
parameters in the Utility Function and challenges around estimation and
measurement in practice. As the utility function is applied twice in the
framework , once in Proposing Selection Technique phase, once in Evaluation
phase, some parameters in the utility function are estimated/measured twice as
well. 

\begin{itemize}
  \item $i$ is an index denoting a particular version $P_i$ of $P$. This seems
  to be always known in any research experiments but this may not be known in
  practice at technique determination time. 
  \item $k$ is an index denoting a particular technique $M_k$ of $M$. This is
  the number of technique implemented in the framework and this is a known fact.
  \item $CA_{in}(i)$ is the cost of instrument all units in $P_i$.
  \item $CA_{tr}(i)$ is the cost of collecting traces for test cases in
  $T_{i-1}$ over $P_{i-1}$ 
  \item $CR(i)$ is the cost of applying test selection technique $M_i$
  \item $CE_{avg}(i)$ is the average execution cost of one test case in $T_i$
  \item $CV_{avg}(i)$ is the average test result validation cost of one test
  case, including automated and manual tests
  \item $CF$ is the cost of missing faults that the regression test suite
  could have detected if executed in full.
  \item $\theta_c$ ratio of cost in regression phase(critical phase) over
  pre-regression phase (preliminary phase)
  \item $\theta_p$ ratio of cost in post-release phase over pre-regression phase
  (preliminary phase)
\end{itemize}


\subsection{When do you want to disable cost model in our framework}
In previous sections, we describe a cost model which is the base of the utility
function in our framework and provided some guidelines about estimation and measurement of the parameters
in the utility function. Later in chapter \ref{chap:Empirical Evaluations}, we
will discuss values of these parameters used in our empirical evaluations.
However, for test practitioners to use the framework in the industry, we
propose disable the utility function in the framework. In the
following section, we will elaborate our reasons.

In practice, even a simplified cost model as we described in
section~\ref{Utility_function} can not be put into use in a single step. It must
be refined and improved iteratively over time. For example, the COCOMO
model \cite{boehm2000software} has gone through several calibration and
evaluation steps, and it took 15 years over 63 projects to improve model's
accuracy. Years of calibration and evaluation is a significant investment for
any organization. Cost model calibration requires estimating the parameters and applying the model at technique determination phase, then tracking and calculating the actual values at evaluation phase for every projects within the organization. 
This would be a significant overhead to the product development and testing and
it could be the major road block in integrating software test selection into testing
processes.From overall cost-effective point of view, there may not be any
benefit until an effective cost model is built.
Also as software development practice and software product evolves so rapidly in the industry, history data collected in previous
project may become obsolete in just number of years to even a few months. Few
companies would be able to afford 15 years to fine tune a cost model.In the end, the return of building an accurate cost model
to determine the ``best'' test selection technique may not even cover the cost
of years of calibration and evaluation. Confronting between dilemma of building
an accurate cost model to determine the ``best'' technique or estimating a
``good'' technique to apply to right away, many organization in the industry
would choose the latter. 

In general, building the utility function for an organization requires 
expert knowledge in cost modelling, manual efforts to collect and enter
parameters, and significant cost to get it right. Since one of the major challenges for industry to adopt test selection techniques is lack of
user friendly tools\cite{4659253}, we suggest organizations disabling
the utility function in the framework, so that the framework becomes an automated and easy to use tool with
low setup cost and minimum manual intervention. After organizations pass the
initial hurdles and adopt the automated test selection practice, they may enable
the utility function to get greater benefits from test selection techniques.

 
Organizations may decided not to fix a bug in a particular release (i.e add the bug into a known issue list) but it makes no sense not to discover a bug in
testing phase. therefore, from practically point of view, test teams would
lalways favour on safe techniques over unsafe technique. 


  
%----------------------------------------------------------------------
% END MATERIAL
%----------------------------------------------------------------------
% B I B L I O G R A P H Y
% -----------------------

% The following statement selects the style to use for references.  It controls the sort order of the entries in the bibliography and also the formatting for the in-text labels.
%\bibliographystyle{plain}
% This specifies the location of the file containing the bibliographic information.  
% It assumes you're using BibTeX (if not, why not?).
%\bibliography{../../bib/thesis}
% The following statement causes the title "References" to be used for the biliography section:
%\renewcommand{\bibname}{References}
%\addcontentsline{toc}{chapter}{\textbf{References}}
% Tip 5: You can create multiple .bib files to organize your references. 
% Just list them all in the \bibliogaphy command, separated by commas (no spaces).

% The following statement causes the specified references to be added to the bibliography% even if they were not 
% cited in the text. The asterisk is a wildcard that causes all entries in the bibliographic database to be included (optional).
%\nocite{*}




% \include{chapters/chapter7/chapter7}
%% UW Thesis Template for LaTeX 
% Last Updated July 8, 2009 by Stephen Carr, IST Client Services
% FOR ASSISTANCE, please send mail to rt-IST-CSmathsci@ist.uwaterloo.ca

% Effective October 2006, the University of Waterloo 
% requires electronic thesis submission. See the UW thesis regulations at
% http://www.grad.uwaterloo.ca/Thesis_Regs/thesistofc.asp.
% However, many faculties/departments also require one or more printed
% copies. This template attempts to satisfy requirements for both types of output. 
% It is based on the standard "report" document class. The "book" document 
% class can be substituted if you have a very large, multi-part thesis.

% DISCLAIMER
% To the best of our knowledge, this template satisfies the current UW requirements.
% However, it is your responsibility to assure that you have met all 
% requirements of the University and your particular department.
% Many thanks to Colin Alie for assistance in preparing this updated template.

% -----------------------------------------------------------------------

% By default, output is produced that is geared toward generating a PDF 
% version optimized for viewing on an electronic display, including 
% hyperlinks within the PDF.
 
% E.g. to process a thesis called "mythesis.tex" based on this template, run:

% pdflatex mythesis	-- first pass of the pdflatex processor
% bibtex mythesis	-- generates bibliography from .bib data file(s) 
% pdflatex mythesis	-- fixes cross-references, bibliographic references, etc
% pdflatex mythesis	-- fixes cross-references, bibliographic references, etc

% If you use the recommended LaTeX editor, Texmaker, you would open the mythesis.tex
% file, then click the pdflatex button. Then run BibTeX (under the Tools menu).
% Then click the pdflatex button two more times. If you have an index as well,
% you'll need to run MakeIndex from the Tools menu as well, before running pdflatex
% the last two times.

% N.B. The "pdftex" program allows graphics in the following formats to be
% included with the "\includegraphics" command: PNG, PDF, JPEG, TIFF
% Tip 1: Generate your figures and photos in the size you want them to appear
% in your thesis, rather than scaling them with \includegraphics options.
% Tip 2: Any drawings you do should be in scalable vector graphic formats:
% SVG, PNG, WMF, EPS and then converted to PNG or PDF, so they are scalable in
% the final PDF as well.
% Tip 3: Photographs should be cropped and compressed so as not to be too large.

% To create a PDF output that is optimized for double-sided printing: 
%
% 1) comment-out the \documentclass statement in the preamble below, and
% un-comment the second \documentclass line.
%
% 2) change the value assigned below to the boolean variable
% "ElectronicVersion" from "true" to "false".

% --------------------- Start of Document Preamble -----------------------

% Specify the document class, default style attributes, and page dimensions
% For hyperlinked PDF, suitable for viewing on a computer, use this:
%----------------------------------------------------------------------
% MAIN BODY
%----------------------------------------------------------------------
% Because this is a short document, and to reduce the number of files
% needed for this template, the chapters are not separate
% documents as suggested above, but you get the idea. If they were
% separate documents, they would each start with the \chapter command, i.e, 
% do not contain \documentclass or \begin{document} and \end{document} commands.
%======================================================================
\reversemarginpar
\setlength{\marginparwidth}{2cm}
\chapter{Empirical Evaluations}
\label{chap:Empirical Evaluations}
%======================================================================
\newpage
%----------------------------------------------------------------------
\section{A Test Selection Framework}
%----------------------------------------------------------------------

\underline{talk about implementation of the framework}
We have implemented 3 Test Selection techniques selected from literature.
[Rothermel and Harrold 1994] technique requires following steps:
1. CFG of the program- G. We utilize BCEL and FINDBUGS API to build CFG of each method of program p and p'. This is implmented in uw.star.sts.analysis package CFGFactory class . //TODO: more description about BCEL and FINDBUGS here..
2. Code Instrumentation.  We use EMMA code coverage tool to collect code coverage information. We have modified test exeuction scripts so that an  Emma  XML report is produced after execution of each test case. We then parse the XML report to get code coverage of each method and each class of program p and p'. Due to limitation in Emma, we can not get branch trace and edge trace information. An edge trace is a trace matrix between edge(n1,n2) in G test case t in T.  A trace is set in the matrix if and only if , when P is executed with t, statements in basic block n1 and basic block n2 are executed sequentially at least once. Edge trace is repesented as a Trace object in uw.star.sts.artifact.Trace class. 
We then investigate several code coverage and code analysis tools. the main goal
is to collect code coverage information not in terms of how many statements were
covered by execution of a test case, but which statements were covered . Many
code coverage tools report accumulated statement coverage information only.
(i.e. #of statements covered). We invetigated Emma,CodeCover,
3.function TestOnEdge(n1,n2) - given an Edge from Basic Block n1 to Basic Block n2, this function returns all tests in T that traversed edge(n1,n2) in G.
4.Intraprocedure SelectTests algorithm - 


implmentation choice #2, [Vokolos,Fankl]
2. code ocverage. build code coverage information in a Trace matrix between
every test case and every statement in all source files. Statement is uniquely
identified with sourcefileName.lineNumber. This is achieved by using CodeCover.

Questions about codecover:
what is a statement coverage? is it on source code statement or bytecode?

How is Html report generated with colour coding? read from which input files?
Answer: codecover has the statment in HTML file and coverage type. so I could
just parse class and linenumber in html file
<span class="covered fullyCovered
Strict_Condition_Coverage">model.isFileModified(fileId)</span>

In Emma - 
Source lines containing executable code get the following color code:

green for fully covered lines,
yellow for partly covered lines (some instructions or branches missed) and
red for lines that have not been executed at all.
Lines without any executable code have no colour at all.

a covered line has the following html code
<tr class="c"><td
class="l">88</td><td>&nbsp;&nbsp;&nbsp;static&nbsp;org.apache.log4j.Category&nbsp;cat&nbsp;=</td></tr>

http://emma.sourceforge.net/coverage_sample_b/_files/5.html
a partial covered line
<tr class="p"><td class="l" title="8% line coverage (1 out of 13
% instructions)"><a name="6">111</a></td><td title="8% line coverage (1 out of 13 instructions)">&nbsp;&nbsp;&nbsp;&nbsp;&nbsp;&nbsp;&nbsp;&nbsp;}</td></tr>



Can I parse the same input files in
a different way to generage the code coverage matrix?
 

\newpage

\TODO{add diagram here goal-data-metrics-evaluation metrics etc. on whiteboard}
\subsection{Identified test selection goals}

\subsection{summary}
Some selection goals and techniques are also come from the knowledge in the
industry. As this framework is used in practice, test engineers can add
additional goals and techniques to the catalogues in the framework. Knowledge
and experience from wide range of industry settings can be recorded in the
framework and made available for sharing. Test engineers and researcher would
have access to selection goals and techniques beyond their immediate domain ,
and can adapt them to meet the needs for particular domain or project.

\subsection{Evaluation Criteria}
Precision is the percentages of test cases selected from previous version for
regression testing. Lower precision number indicates greater savings which is
better.Since test cases new to this version are always need to be executed
anyways,they are not counted in the precision calculation.Formally, we define
precision p as (|T'|/|T|)*100 
    T' - test cases selected from previous version (should not include new test
   cases in this version) 
   T - test cases in previous version

In addition, some of the goals are ambigous or not granaluar enough. 
\subsection{questions}
The goal of the evaluation leads to the following questions:
Does using the STS framework improve the probability of selecting the most
appropriate techniques? 
Does the additional cost of selecting technique make the test selection
in-effective?
\todo{compare against challenges}

\subsection{quantitative analysis}

\subsection{qualitative analysis}
In our evaluation, all tests are junit test case which in average takes about
1~2 seconds to run. The maximum test suite size among our evaluation test
subjects is 500 test cases(jboss). Even if there were a STS technique selecting
just one magic test case out of the entire test suite, the saving in time would
no more than 16 minutes. This is calculated by 500 test cases * 2 seconds /60.
It's unlikely to have a cost effective technique for Junit test suites of this
size. Even if the technique is cost effective, the saving is not significant
and the difference in savings between techniques is trivial.

However, in industry settings, the size of the test suite, the average cost of
executing a test case and the total number of version could be in completely
different magnitude. For example, L White et al. conducted an experiement on a
ABB properitory software.\cite{DBLP:journals/smr/WhiteJRR08} For version 1, 88
test cases take 15.7 hours to run. The average test execution time is 642
seconds, which is 300 times more than our evaluation test subjects.
<todo:examples of number of versions and number of test cases as well.> 


Also , as we discussed in <cost model>, regression test cost also consists of
<some other cost factors, that is significantly higher than in our experiement
settings>



future works:

SIR repository - 
Folder struture problems:
SIR repository is a collection of many test subjects and provides
valuable data for research. However, the repository is created and maitentained
manually and mostly for use manually. Those test subjects are organized in a
predefined structure, they are not consisitent and accurate enough for program
to use.  e.g. versions directory exist in some test subjects but not exist in
others. compile class files sometimes placed under build/classes, sometime
placed in build/ant/classes (various by versions). These difference between test
subjects and between different versions of the same subject may be tolleranted
by a human user, but will not be accepted by a automated program. 

Test execution scripts problem
in jmeter, verion5, test case org.apache.jmeter.protocol.http.sampler.PackageTest
exist in test plan (v5.class.junit.universal.all file) but not in test
exeuction script. (scriptR5.cls).
Problems with SIR:
Many test cases are missing from test execution scripts:jakata-jmeter
classpath are set with hardcoded absolute path, should set with variables
 
Missing debug information in compiled classes:
In some test subjects, source code are already compiled but not compiled with
full debug data ( ref java JFC) It's cruical for most of the
analysis tools to have the full debug data in order to analyze the code
coverage information. For example, for Emma to work, java progams have to be
compiled with option debug='true' debuglevel = ``'' in ant.
Missing debug data causes many additional time to recompile each version of the
test subjects and to resolve compile issues of each version (due to depency
libaries, deprecated methods etc.) Ideally, each test subject should be availabe
to use without any knowledge about the test subject itself- in term of how to
compile it. For propritery software, source code may not even avaiable to
modify!

code is written with old verions of J2SE (1.3)
there are 51 instance of ``enum' used as a varialble name, while enum became a
keyword since java 1.5. So manual updates are needed to change all 51 instance
of enum to _enum in all 8 versions. This is extensive amount of manul work and
this problem is exposed by previous problem (having to recompile). Had all test
subjects are compiled with debug information, recompile & manul edit wouldn't
have been necessary. 


The framework requires a unified ALM tool to manage all artifacts in the SDLS
and automaticallly maitain the traceability. This is largely a solved problem in
the industry as there are many commerical ALM tool set avaiable. However, in the
research community, commerical tool is either  too expensive or too heavy to
use. Future work to consolidate all SIR test subjects to one light weight,
free/open source ALM tool.


Evaluation with industry data. 

%----------------------------------------------------------------------
% END MATERIAL
%----------------------------------------------------------------------
% B I B L I O G R A P H Y
% -----------------------

% The following statement selects the style to use for references.  It controls the sort order of the entries in the bibliography and also the formatting for the in-text labels.
%\bibliographystyle{plain}
% This specifies the location of the file containing the bibliographic information.  
% It assumes you're using BibTeX (if not, why not?).
%\bibliography{../../bib/thesis}
% The following statement causes the title "References" to be used for the biliography section:
%\renewcommand{\bibname}{References}
%\addcontentsline{toc}{chapter}{\textbf{References}}
% Tip 5: You can create multiple .bib files to organize your references. 
% Just list them all in the \bibliogaphy command, separated by commas (no spaces).

% The following statement causes the specified references to be added to the bibliography% even if they were not 
% cited in the text. The asterisk is a wildcard that causes all entries in the bibliographic database to be included (optional).
%\nocite{*}


\end{document}


% The \appendix statement indicates the beginning of the appendices.
\appendix

% Add a title page before the appendices and a line in the Table of Contents
\chapter*{APPENDICES}
\addcontentsline{toc}{chapter}{APPENDICES}
%======================================================================
\chapter[PDF Plots From Matlab]{Matlab Code for Making a PDF Plot}
\label{AppendixA}
% Tip 4: Example of how to get a shorter chapter title for the Table of Contents 
%======================================================================
\section{Using the GUI}
Properties of Matab plots can be adjusted from the plot window via a graphical interface. Under the Desktop menu in the Figure window, select the Property Editor. You may also want to check the Plot Browser and Figure Palette for more tools. To adjust properties of the axes, look under the Edit menu and select Axes Properties.

To set the figure size and to save as PDF or other file formats, click the Export Setup button in the figure Property Editor.

\section{From the Command Line} 
All figure properties can also be manipulated from the command line. Here's an example: 
\begin{verbatim}
x=[0:0.1:pi];
hold on % Plot multiple traces on one figure
plot(x,sin(x))
plot(x,cos(x),'--r')
plot(x,tan(x),'.-g')
title('Some Trig Functions Over 0 to \pi') % Note LaTeX markup!
legend('{\it sin}(x)','{\it cos}(x)','{\it tan}(x)')
hold off
set(gca,'Ylim',[-3 3]) % Adjust Y limits of "current axes"
set(gcf,'Units','inches') % Set figure size units of "current figure"
set(gcf,'Position',[0,0,6,4]) % Set figure width (6 in.) and height (4 in.)
cd n:\thesis\plots % Select where to save
print -dpdf plot.pdf % Save as PDF
\end{verbatim}

%----------------------------------------------------------------------
% END MATERIAL
%----------------------------------------------------------------------

% B I B L I O G R A P H Y
% -----------------------

% The following statement selects the style to use for references.  It controls the sort order of the entries in the bibliography and also the formatting for the in-text labels.
\bibliographystyle{plain}
% This specifies the location of the file containing the bibliographic information.  
% It assumes you're using BibTeX (if not, why not?).
\cleardoublepage % This is needed if the book class is used, to place the anchor in the correct page,
                 % because the bibliography will start on its own page.
                 % Use \clearpage instead if the document class uses the "oneside" argument
\phantomsection  % With hyperref package, enables hyperlinking from the table of contents to bibliography             
% The following statement causes the title "References" to be used for the bibliography section:
\renewcommand*{\bibname}{References}

% Add the References to the Table of Contents
\addcontentsline{toc}{chapter}{\textbf{References}}

\bibliography{bib/mythesisbib}
% Tip 5: You can create multiple .bib files to organize your references. 
% Just list them all in the \bibliogaphy command, separated by commas (no spaces).

% The following statement causes the specified references to be added to the bibliography% even if they were not 
% cited in the text. The asterisk is a wildcard that causes all entries in the bibliographic database to be included (optional).
\nocite{*}

\end{document}
